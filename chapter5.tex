\chapter{General Discussion and Conclusion}

The police have a broad mission in society which includes responding to public health issues such as opioid overdoses. As opioid overdose mortality has increased over the last two decades, the police have adapted to be more effective in reducing opioid overdose fatalities by carrying naloxone and, in some cases, collaborating with public health agencies to connect individuals to services \parencite{formica_characteristics_2021, ray_national_2023}. However, there are debates about the role of the police in responding to opioid overdoses. The debates center around the role of the police as law enforcers leading to criminalizing PWUO or others on scene \parencite{carroll_police_2023,lowder_twoyear_2020, bohnert_policing_2011}, which can lead to a fear of calling 911 when an overdose takes place \parencite{van_der_meulen_thats_2021}; and iatrogensis, such as drug seizures leading to an increase in opioid overdoses in surrounding areas \parencite{ray_spatiotemporal_2023}. It is the broad debate about the role of the police that this dissertation set out to inform. This dissertation made 3 primary contributions as it 1) investigated when and where police were more likely to administer naloxone at an opioid overdose, 2) examined the compassion fatigue hypothesis among police officers and their perceptions towards their role in responding to opioid overdoses, naloxone, and PWUO, and 3) provided an evaluation of the impact of a police-public health collaborative post-overdose outreach program (Tempe First-Responder Opioid Recovery Project; ORP). Thus, the three studies highlight when and where police may be able to reduce harm, their perceptions of responding to opioid overdoses, naloxone, and PWUO, and if a police-led collaborative post-overdose outreach program reduced opioid overdoses and fatal opioid overdoses.

Specifically, Study 1 used spatial point pattern tests to assess if there was spatial variation in where police administer naloxone first at an opioid overdose, compared to Tempe Fire and Medical Rescue (TFMR). Then, to investigate why spatial variation exists, mixed-effects regression models were used to examine the incident- and neighborhood-level characteristics that are associated with police responding first to an opioid overdose and administering naloxone. Study 2 examined officer attitudes towards their role in responding to opioid overdoses, naloxone, and PWUO over 4 waves of surveys. The wave of the survey was used as a measure of time, and opioid overdose response frequency was used as a proxy for exposure to vicarious trauma to test the compassion fatigue hypothesis. Lastly, study 3 employs a comparative interrupted time series model comparing Tempe's opioid overdose calls for service rate and fatal opioid overdose rate to an adjacent city, Mesa. The rest of this chapter discusses the main findings from each study, the theoretical and policy implications, and directions for future research.

\section{\centering Summary of Main Findings}
\subsection{Study 1}
Study 1 reports a few important findings. First, there is spatial variation in where police were administering naloxone first compared to Tempe Fire and Medical Rescue. The \textit{S}-index suggests that there is an overlap of .336 (robust) to .387 (standard), suggesting that there is spatial dissimilarity between where the two first responders are administering naloxone first. Second, in examining why this variation exists, both neighborhood- and incident-level characteristics were associated with police administering naloxone first. Neighborhoods explain approximately 10\% and 1\% of the variation for police administering naloxone first and TFMR administering naloxone first, respectively. 

Police officers administering naloxone first at opioid overdose incidents was less likely when the incident was originally dispatched as a health/medical related issue (31\% decreased odds), mental health issue (63\% decreased odds), and other (36\% decreased odds), compared to a call dispatched as an overdose. Police officers were also less likely to administer naloxone first when the individual was older, specifically for those aged 40 - 59 (39\% decreased odds) and 60 - 99 (60\% decreased odds), compared to those aged 20 - 29. At the neighborhood-level, police officers were more likely to administer naloxone in neighborhoods that had higher drug offense rates (.001\% increased odds), higher proportion of White (.023\% increased odds) and Black (.029\% increased odds) residents, and were less likely to administer naloxone first a in neighborhoods with a higher proportion of owner occupied units (.037\% decreased odds). Police officers were also less likely to administer naloxone first in neighborhoods that had higher violent crime rates (.004\% decreased odds).

When comparing police and TFMR directly, the violent crime rate, drug offense rate, percentage Black, mental health related calls for service, and other calls for service predict variation in when and where police and TFMR respond first to administer naloxone. Specifically, compared to TFMR, police officers were more likely to administer naloxone first in neighborhoods with higher drug offense rates, percentage Black residents, and were less likely to administer naloxone in neighborhoods with higher violent crime rates, when the incident was dispatched as mental health related or other compared to an overdose, and when the individual was aged 40 to 59 (see Figure 2).

\subsection{Study 2}
Study 2 provides three important findings. First, when assessing officer attitudes across the waves -- using one-way fixed effects models -- officers are more supportive of their role in the opioid overdose crisis in wave 4, compared to wave 1. Second, officers develop more negative attitudes towards PWUO in waves 3 and 4, compared to wave 1. Third, opioid overdose response frequency was not associated with any of the outcomes in the ordinary least squares (OLS) regression models.\footnote{The results are generally robust to alternative methods such as a subgroup analysis of patrol officers only and weighting the data. When using wave specific weights, responding multiple times per shift to an opioid overdose is associated with an increase in naloxone risk-related compensation beliefs in the univariate model but not in the multivariate model (see Appendix B, Table 16). Also, patrol officers seem to not develop negative attitudes towards PWUO across waves (see Appendix B, Table 17). Moreover, Appendix B, Table 19 treats the wave of the survey as a continuous measure of time and the findings reinforce the main fixed effects models.} This finding suggests that compassion fatigue --modeled with opioid overdose response frequency as the predictor-- is not present among Tempe police officers. The findings collectively speak to the possibility that police officers in Tempe believe that they should have a hand in responding to and administering naloxone to preserve life, but at the same time, exhibit more negative attitudes towards PWUO over time. 

\subsection{Study 3}
Study 3 compares Tempe's opioid overdose calls for service rate and fatal opioid overdose rate to Mesa's from May of 2018 through October of 2021. With the Tempe ORP beginning in February 2020, the intervention was not associated with a significant reduction in the opioid overdose calls for service rate or fatal opioid overdose rate in the post-intervention period, compared to Mesa's trends. The null findings suggest that in the aggregate, the project did not have an impact. 

\subsection{Collective Takeaway}
Together, the three studies broadly inform the debate about the role of the police in the opioid overdose crisis. Specifically, studies 1 and 2 highlight important antecedents of developing an effective police-led collaborative opioid overdose outreach program. Namely, the two studies address two research gaps on when and where police officers are administering naloxone and if officers experience compassion fatigue. The former is important for identifying where police -- and other first responders -- can be directed to connect individuals to services, while the latter is vital for program ``buy-in" and sustainability \parencite{winstanley_bell_2020}. Lastly, study 3 speaks to the effectiveness of a police-led collaborative opioid overdose outreach program in Tempe, Arizona. Though the comparative interrupted time series analysis yielded null findings, there were important indicators of program success over the course of the four year project. For example, over 294 naloxone administrations for Tempe police officers and a higher engagement rate than similar programs \parencite{watts_tempe_2023}, which suggests that police officers can play an critical role in providing life-saving care and initiating a post-overdose outreach response.

\section{Theoretical and Policy Implications}
\subsection{Theoretical Implications}
Study 1 offers a few theoretical contributions worth discussing. First, broadly speaking, the findings expand on prior work examining the variation in police and EMS data to explain why the variation exists (see \cite{hibdon_use_2024}). Although the present study used one dataset, being able to identify when police were on scene first allowed for an analysis between police and TFMR involvement in responding to opioid overdoses. The findings suggest that police are on scene first at an opioid overdose in different neighborhoods than TFMR. Specifically, police officers are more likely to administer naloxone first at an opioid overdose in neighborhoods that are comprised of higher drug offense rates and higher proportion of Black residents. Police officers are less likely to administer naloxone first at an opioid overdose in neighborhoods with higher violent crime rates.

Moreover, the findings can be viewed through the framework of \citeauthor{klinger_negotiating_1997}'s \citeyear{klinger_negotiating_1997} ecological theory of police work. For instance, the relationship between violent crime rates and police being less likely to be first on scene might suggest that the majority of officers' time is spent on crime-related calls positioning them to not be the first-responder at an opioid overdose. Granted, this only speaks to part of Klinger's theoretical framework. \textcite{klinger_negotiating_1997} also suggests that this slow response would due to cynical views of the community members in high crime neighborhoods, however, the present study could not speak to the perceptions of police officers towards these neighborhoods. 

Additionally, police officers being more likely to administer naloxone first in neighborhoods with a higher proportion of Black residents suggest that officers are situated in and around these neighborhoods to respond first. This is particularly interesting because the percentage of Black residents and opioid overdose rates are negatively correlated, at the neighborhood level. There are two potential reasons for this: 1) police resources are distributed in and around neighborhoods with a higher proportion of Black residents, or 2) TFMR's response times are delayed for these neighborhoods. It could also be a combination of the two. Nonetheless, \textcite{klinger_negotiating_1997}, and other conflict perspectives \parencite{black_manners_1980}, would hypothesize that police would have slower response times in marginalized communities, which is not the case in Tempe. Police officers are responding quickly to administer naloxone at an opioid overdose.

\subsection{Policy Implications}
Study 1 highlights which incident and neighborhood characteristics are associated with police officers being more likely to respond first at an opioid overdose and administer naloxone. First, the findings suggest that police officers in neighborhoods with higher percentage of White and Black residents, drug offense rates, as well as more renters and vacant units, are outfitted with naloxone and that they have access to replenish their naloxone doses. Second, these findings also suggest that these neighborhoods are most likely for police officers to connect individuals to services through the 24/7 hour hotline to contact EMPACT. This could identify areas where both police and EMPACT can be proactive in their efforts to distribute naloxone and locate individuals who have overdosed to continue contact, build rapport, and provide information related to the services available through the project. Lastly, in neighborhoods with higher violent crime rates which is associated with police not responding first, coordination and collaboration between police and TFMR could aid in providing timely responses to overdoses in these neighborhoods. If police response is faster, yet they are unable to response in a timely manner because their time is spent on crime-related calls, TFMR should be aware of this to devote attention to these areas. Another possibility is that there is a specialized unit within the police department or a co-response type of model to focus directly in these neighborhoods where responses may be delayed. 

Study 2 found that officers believe they have a role to play in saving lives in the opioid overdose crisis, which suggest there is ``buy-in" to the Tempe ORP which is vital for program success (see \cite{winstanley_bell_2020}). However, there is an increase in negative attitudes towards PWUO across the survey waves. Interestingly, this trend across the waves seems to be concentrated among non-patrol officers. An important step is to have periodic refresher trainings on opioid use. Prior work has demonstrated that trainings can improve officers perceptions on fentanyl myths, naloxone risk-compensation beliefs, and competence \parencite{del_pozo_can_2021, wagner_training_2016, white_narcan_2021, winograd_concerns_2019}. Training's on opioid use and addiction may combat the rise of negative attitudes towards PWUO.

Lastly, the findings from study 3 suggest that in the aggregate the Tempe ORP did not impact opioid overdose calls for service or fatal opioid overdoses. Despite the null findings, police officers in Tempe view themselves as having a role in the opioid overdose crisis (see study 2 and \cite{white_narcan_2021, white_moving_2021}), have administered naloxone over 250 times since February 2020 with a high success rate (91\% effective), and have contacted the post-overdose outreach response approximately the same number of times \parencite{watts_tempe_2023}. Moreover, they have exercised their discretion at the scene of opioid overdoses, infrequently arresting overdose victims or others on scene \parencite{white_leveraging_2022}, prioritizing linking individuals to La Frontera EMPACT for services. This has led to high engagement rates (approximately 50\%) between EMPACT and individuals who have previously overdosed, which is higher than prior research in this area has indicated \parencite{dahlem_recovery_2021, wagner_training_2016}. These are without a doubt important outputs that indicate program success. 

However, the reach of police officers at opioid overdoses is limited because they are responding to a subset of the total call volume of opioid overdoses. This likely hinders the programs effect at the city-level. However, at a program-level, there may be an effect between those who accepted services and did not. Also, folding in TFMR to be able to to initiate the post-overdose outreach would likely yield larger effects in the aggregate. Of course, however, this would raise the cost of the program. Other jurisdictions interested in post-overdose outreach programs, and experience similar variation in how many opioid overdose incidents police are on scene at opioid overdoses, may want to expand the program to include both EMS and police, if feasible.

Lastly, data collection, management, and dissemination practices for opioid overdose data must improve to facilitate timely and rigorous evaluations of city-wide interventions. Improvements in these areas can help limit the methodological constraints that researchers face. Indeed, comprehensive city-level data availability could facilitate robust quasi-experimental methods to better understand city-level interventions which would ideally lead to more effective policies and programs being implemented. Without this data, it is difficult -- if not impossible -- to produce reliable estimates of treatment effects.

\section{\centering Directions for Future Research}
The findings from study 1 suggest there should be more attention paid to why variation between police and EMS data exists. With the growing body of research that investigates the overlap and concentration of police and EMS data \parencite{hibdon_use_2024}, it would be useful going forward to examine the context that produces this variation. As mentioned above, this could provide local agencies with actionable information to improve responses to areas where response times are delayed. Moreover, because variation exists, this could impact how a police-led program should be evaluated. The effect of a police-led program may be unequal across a given jurisdiction with some areas being exposed to the project more than others. This will be discussed more below. Additionally, as \textcite{lowder_twoyear_2020} note, understanding when and where individuals are more likely to be arrested at an opioid overdose is an important avenue. Study 1 contextualized when and where police arrive first on scene at an opioid overdose to administer naloxone but is not linked to incident level arrest data. Linking these data sources would be key for a more thorough understanding of when and where police are responding to and exercising their authority to arrest an opioid overdose victim (see \cite{lowder_twoyear_2020}). Lastly, qualitative work on officer perceptions of the neighborhoods they respond to and administer naloxone would be useful to contextualize some of the findings. This would contribute to a better understanding of how \citeauthor{klinger_negotiating_1997}'s (\citeyear{klinger_negotiating_1997}) ecological theory of police work may apply to police officers responding to opioid overdoses in high violent crime neighborhoods and in neighborhoods with a higher proportion of Black residents.

Related to officers perceptions towards their role in the opioid overdose crisis, naloxone, and PWUO, and the compassion fatigue hypothesis, future research should take a few different avenues. First, the findings from study 2 investigate aggregate mean variation across survey waves. While informative, future research should expand on this to examine within-officer variation over time. Ideally, this would entail a panel dataset of officers attitudes over time. Second, and somewhat related to the first point, investigating how attitudes impact officer behavior could shed light on how, for instance, negative attitudes towards PWUO effect officer decision-making. Increasingly negative attitudes towards PWUO could result in more punitiveness towards this population. Currently, it is not clear if this relationship exists but it could have detrimental effects for the overdose victim (e.g., arrest) and the effectiveness of a collaborative opioid overdose program (e.g., interruption of connection to services/treatment). Lastly, employing a validated compassion fatigue scale within the policing context would be useful to more accurately capture this latent construct (see \cite{adams_compassion_2006}). Prior research has primarily used opioid overdose response frequency as a proxy for exposure to trauma. Study 2 took the same approach but also included a measure of time to capture how compassion fatigue may compound over time. Therefore, future research in this area should be more precise in their measurement of compassion fatigue.

Additionally, study 3's null findings and the discussion provide a path for future research. First, because police officers are a) on scene at an opioid overdose in Tempe for approximately 40\% of the total call volume to opioid overdoses that TFMR receives and b) are more likely to arrive on scene and administer naloxone first at opioid overdose in certain neighborhoods (see study 1), it may be the case that the ORP's effect was at a more granular level. Recall that the initiation into the project does not occur unless police are on scene. Thus, the effect may be negligible in the aggregate. If this variation exists in other cities, it may be useful to consider how the program impacted certain subgroups, such as specific neighborhoods, more so than others. Second, related to the first point, it would be useful to compare outcomes among those who accepted services and those that did not, using matching techniques to create balance between treatment and control (see \cite{perrone_harm_2022}). Lastly, to reach more of the opioid using population, it would be important to involve both police and EMS in the initiation of a post-overdose outreach, particularly if police only on scene for a subset of the total opioid overdose call volume. In the aggregate, this would likely have a larger effect.

\section{\centering Conclusion}
This dissertation set out to provide a broad account of police involvement in the opioid overdose crisis to inform the debates revolving around the role of the police in responding to opioid overdoses. There were three primary objectives of this dissertation. First, investigate the incident- and neighborhood-level characteristics that are associated with police responding first to an opioid overdose and administering naloxone. Second, investigate the compassion fatigue hypothesis by examining officers' attitudes towards their role in the opioid overdose crisis, naloxone, and PWUO across multiple survey waves. Third, and finally, evaluate the Tempe First-Responder Opioid Recovery Project to understand its impact on opioid overdose calls for service rates and fatal opioid overdose rates. The results from this dissertation highlight the complexity of police involvement in the opioid overdose crisis. Namely, police officers are situated to respond first to opioid overdoses in different neighborhoods than Tempe Fire and Medical Rescue, suggesting that they can provide timely life-saving care for different populations than TFMR. Second, police officers generally view themselves as having a role in the opioid overdose crisis, yet there seems to be an increase in negative attitudes towards PWUO over time. And lastly, the Tempe ORP did not impact opioid overdose calls for service rates or fatal opioid overdose rates at the city-level. This dissertation informs theory and policy focused on how the police can be utilized in the opioid overdose crisis. As \textcite{monaghan_broken_2022} states, ``The immense cost of the opioid epidemic, and the failure of pure law enforcement strategies, suggest that not only ought police to carry naloxone, but that police should also adopt a wide range of harm reduction strategies" (pg. 326). The Tempe First-Responder Opioid Recovery Project embodies this quote and demonstrates an innovative approach to reducing harm in the opioid overdose crisis. 








