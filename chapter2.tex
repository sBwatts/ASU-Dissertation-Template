\chapter{Investigating the Intersection of Policing, Place, and Opioids}


Historical and Theoretical Foundations of the Importance of Place and its Impact on Police

Police work is often tied to space and time (Fyfe, 1991). Police patrol and activity often reflects local crime trends and areas marked by disadvantage. Across cities, crime – particularly violent crime – tends to be concentrated in neighborhoods characterized by socioeconomic disadvantage (Peterson & Krivo, 2010). And the distribution of disadvantage and crime in a city is not random (see Rothstein, 2017; Weisburd, 2015) which highlights the spatial concentration of criminal activity. The theoretical underpinnings that highlight how crime is spatially concentrated are important to consider given its implications for the police.

Social Disorganization Theory

Early geographical work of the 1800s inspired the contemporary Chicago school which emerged in the early 20th century. The Chicago school is a sociological perspective that explains how the structure of urban life influences outcomes. In one of the earliest tests of the role of urban structure and crime, Shaw & McKay (1942) investigated why delinquency was unevenly distributed throughout the city. They found that areas that were socially disorganized – high levels of poverty, racial heterogeneity, and residential instability – experienced higher delinquency rates. Shaw and Mckay’s (1942) finding indicated that delinquency was not solely associated with individual-level characteristics, but that structure played a role as well. These findings led to the development of social disorganization theory which suggests that there are neighborhood-level characteristics that influence juvenile delinquency rates regardless of the individual characteristics of those in the neighborhood. One of the primary mechanisms used to explain this finding is that the formation of social ties is hindered in socially disorganized neighborhoods. The systemic model, proposed by (Kasarda & Janowitz, 1974), places an emphasis on the social connections between community members (see Bursik & Grasmick, 1993; Bursik, 2000). Specifically, neighborhoods with strong ties between community members can convey norms, values, and expectations of conduct for the neighborhood. The transmission of these norms and values produces a greater capacity for informal social control among community members (Kornhauser, 1978). However, the presence of disadvantage, residential instability, and ethnic heterogeneity can hinder the development of informal social controls. Because there is an inability to develop strong informal social control mechanisms in neighborhoods marked by disadvantage, deviance and other forms of criminal activity can go unthwarted. In one of the first full tests of social disorganization theory, Sampson & Groves, (1989) showed that there was an indirect effect of neighborhood-level structural characteristics on crime and that the strength of networks among community members mediated the relationship. Contemporary work has continued to support this finding showing the impact of structural characteristics and the role of community ties play across jurisdictions (Bursik, 2000; Levy et al., 2020; Sampson, 2012; Sampson et al., 1997). 

Macro-level Strain

Another ecological theoretical framework for explaining the association between community-level characteristics and crime is the macro-level strain theory proposed by (Agnew, 1999). Agnew suggests that the differential distribution of crime across communities is not solely a product of weak informal social controls but is a product of the strain producing characteristics in these communities that either blocks individuals’ goals, produces a feeling of deprivation, introduces negative stimuli, or removes positive stimuli. There is also a network effect wherein individuals interact with other highly strained individuals that can impact one’s stress. Communities with strain producing characteristics select strained individuals, particularly for those who are facing economic hardships. These individuals move into the deprived community and are confined to the community because of their inability to relocate. This then leads to community-level differences in crime rates because of the heightened likelihood of a criminal response to the strain experienced in the given neighborhood.

Yet, tests of Agnew’s (1999) model are relatively limited, particularly at the neighborhood level. Of the research that has been done at the neighborhood level, findings have highlighted the role of strain and its impact on community crime rates (Antunes & Manasse, 2022; Warner & Fowler, 2003). This research indicates that strain does play a role in the production of crime rates at the neighborhood level. Warner & Fowler (2003) show that the role of informal social controls become statistically non-significant while community-level strain is significantly associated with violence. Antunes et al. (2022) report similar findings in that community level strain, exposure to violence, and youth violence are associated even when controlling for collective efficacy. 

Geography and Opioid Overdoses

The same macro-level process of social disorganization theory and macro-level strain theory that are associated with crime are also associated with opioid overdoses. Community-level characteristics that been found to be associated with overdoses and overdose mortality include higher levels of poverty, ethnic heterogeneity, lower home ownership rates, and low educational attainment (Chichester et al., 2020; Galea, 2003; Hannon & Cuddy, 2006). Ford et al. (2017) show that social disorganization, a lack of social capital, and low social participation are predictive of adolescent opioid misuse. Winstanley et al. (2008) also find the same relationship between social capital and reported opioid dependence. This could suggest that higher levels of social capital indicate greater levels of connectedness with community, family, and school. These findings indicate that stronger ties and social integration is associated with lower levels of reported opioid use. This connectedness likely improves the ability for community members to enact informal social controls and discourage drug use. Feldmeyer et al. (2022) notes that structural factors, which include higher opioid prescription rates, population decline, poor community health, and the loss of manufacturing jobs are associated with higher levels of overdoses. Further, Feldmeyer et al.’s findings are in line with the deaths of despair argument proposed by Case & Deaton (2015, 2017). Specifically, that the same ecological characteristics are associated with overdose deaths and suicides. Feldmeyer et al. extend this to show some overlap with homicides as well. In a more recent article looking at drug overdoses in Passaic County, New Jersey from 2015-2019 at the block-group level, Piza et al. (2023) finds that one of the strongest predictors of drug overdoses is concentrated disadvantage. 

The empirical work investigating the association between structural variables and opioid overdoses suggest that weakened social controls (i.e., lower home ownership rates, population decline, disadvantage) and strain (e.g., loss of employment) influence opioid overdose rates. As discussed above, the formation of ties is important for informal social control. Disadvantage, population decline, and lower homeownership rates are neighborhood characteristics that hinder the formation of social connections which reduces the capacity to develop informal control mechanisms. When informal social control is present, members of a community may discourage opioid use resulting in either less use or potentially less risky use. Also, unemployment and the loss of manufacturing jobs has been found to be associated with elevated opioid overdose rates. This is particularly true for the Appalachian region of the US where many hard labor jobs have become obsolete (McLean, 2016). This association suggests there is a relationship between strain and opioid overdoses. Particularly for communities facing loss of employment opportunities and economic decline. Individuals may be strained because of their inability to achieve monetary goals and are unable to relocate given a variety of reasons including their economic situation. Communities facing economic hardships produce strain and retain strained individuals, which then leads to various forms of coping which could be opioid usage (see Monnat, 2018, 2019). While these findings suggest that policies should be implemented at the macro-level to address disadvantage, economic decline, and job loss to help facilitate the development of informal social control and reduce strain, there is a practical implication of this overlap between macro-level characteristics, crime, and opioid overdoses. Namely, the potential for police to have an important role in reducing harm in the opioid overdose crisis given that they are also unevenly distributed throughout a city in areas of disadvantage, with higher rates of opioids overdoses, and higher rates of crime.

Geography and the Police

Due to disadvantage, calls for service, crime, and public health related issues being spatially concentrated, police are likely to be situated in these areas frequently (Engel et al., 2012; Mitchell & Lynch, 2011). Deploying police resources based on need – defined as volume of calls for service – is the underpinning of the “deployment hypothesis” (but see Alexander, 2010; Beckett & Herbert, 2010). Police departments have attempted to maximize the utility of their limited resources by having officers focus on high-crime areas (Skogan & Frydl, 2004; Weisburd et al., 2003). Police activity being spatially concentrated suggests they are positioned to respond to overdoses in places and for populations that may differ from other first-responders, given their availability and patrolling patterns. This is an important aspect of police work that is often overlooked when thinking about their role in responding to public health issues such as the opioid overdose crisis. It has been shown that the police are reaching different communities than EMS for calls for service pertaining to violence and drug activity (Hibdon et al., 2017; Hibdon et al., 2021). Likewise, they may be able to reach overdose incidents prior to EMS or the fire department which could be indicative of a quicker response time in these areas (Pourtaher et al., 2022; White et al., 2022). Debates about the role of the police responding to opioid overdoses typically embody a philosophical argument or one that focuses on iatrogenic effects of the police involvement (Doe-Simkins et al., 2022; Michaud et al., 2023; Van Der Meulen et al., 2021). However, this spatial component of police work as it relates to the opioid overdose crisis suggests the police may be able to administer naloxone and act as a conduit to services in areas where other first responders may infrequently respond to or are delayed in responding. 

