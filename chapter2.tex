\chapter{Investigating the Intersection of Policing, Place, and Opioids}

\section{\centering Introduction}
As opioid overdoses have increased exponentially over the last two decades, the police have increasingly assumed a more complex role in responding to the problem \parencite{quinn_most_2019, ray_national_2023}. In some cases they are involved in post-overdose outreach \parencite{formica_characteristics_2021,ray_national_2023} acting as a bridge to social services. While this is not outside of the police mission, it is a change from how police have traditionally handled substance use issues through arrests \parencite{cooper_war_2015}. Police officers being increasingly outfitted with naloxone -- an opioid antagonist -- means they can be effective in responding to and reversing opioid overdoses. However, there are concerns around police involvement in opioid overdoses. Namely, \citeauthor{lowder_twoyear_2020}'s (\citeyear{lowder_twoyear_2020}) findings suggest that individuals who overdosed and receive a police-response first (compared to emergency medical services; EMS), were more likely to be arrested. This increased criminalization requires further investigation, as \textcite{lowder_twoyear_2020} note. Particularly, investigating the incident and neighborhood factors that are associated with police being first on scene at an overdose is needed. Thus, it is crucial to understand how a police-first response to an opioid overdose is influenced by the communities that they serve.

Public health issues and crime-related issues such as substance use, suicide, and homicide tend to overlap spatially at the macro-level \parencite{feldmeyer_community_2022}. At micro-places, there is some overlap between police and EMS data (e.g., violence and drug activity) \parencite{hibdon_use_2024}. Additionally, police officers tend to outnumber other first responders \parencite{lurigio_opioid_2018}. Because public safety, namely criminal activity, is a primary responsibility of the police, and because police officers are often actively patrolling and outnumber other first responders, police officers' response ecology differs from other first responders. In theory, the spatial convergence of public health and public safety issues position the police to respond quickly to opioid overdoses and provide immediate and effective life-saving care.

While the spatial overlap of crime and opioid overdoses is fairly well established \parencite{carter_spatial_2019, magee_dual_2022}, it is less clear if the police do in fact respond to opioid overdoses quicker in these neighborhoods. Because other public health issues can also overlap with crime and opioid overdoses, EMS or other first responders may be well suited to respond quickly as well. Additionally, it is possible that police operating in higher crime neighborhoods means that most of their time is spent handling higher priority issues \parencite{klinger_negotiating_1997}. This would suggest that police response times to opioid overdoses may be delayed and other first-responders are better suited to respond first in these areas. Prior work that has looked at where police are more likely to administer naloxone tends to aggregate the total rate or count of naloxone administrations or responses to opioid overdoses for police \textit{and} EMS/Fire. This research suggests that, together, police and EMS naloxone administrations do concentrate geographically \parencite{heavey_descriptive_2018}. Other work that has attempted to investigate spatial variation in responsiveness between first-responders indicates that police officers may be more likely to be first on scene at an overdose in rural areas \parencite{wood_overdose_2021}. However, there are no studies that investigate the spatial variation between first responders at more granular geographic areas like the block group, a proxy for neighborhoods. This is an important avenue as it will shed light on when and where individuals are more likely to receive a police-first response at an opioid overdose.

The present study seeks to address this gap in the literature. Using Tempe Fire and Medical Rescue (TFMR) calls for service data I investigate the spatial variation in police administering naloxone first and TFMR administering naloxone first. I use a spatial point pattern test to first descriptively assess if there is variation in where police and TFMR are administering naloxone first across neighborhoods in Tempe, Arizona. Then, because of the nested structure of the data, I employ mixed-effects regression models to assess the incident- and neighborhood-level predictors of police and TFMR administering naloxone first. The findings have implications for policy that guides police discretion and decision-making at opioid overdoses, reinforces the need for inter-agency communication and collaboration, and the need for community-level programs to expand accessibility of naloxone for community members.

\section{\centering Literature Review}

\subsection{Place, Crime, and Opioid Overdoses}

Early geographical work of the 1800s inspired the contemporary Chicago school which emerged in the early 20th century. The Chicago school is a sociological perspective that explains how the structure of urban life influences outcomes. In one of the earliest tests of the role of urban structure and crime, \textcite{shaw_juvenile_1942} investigated why delinquency was unevenly distributed throughout the city. They found that neighborhoods that were socially disorganized – high levels of poverty, racial heterogeneity, and residential instability – experienced higher delinquency rates. \textcite{shaw_juvenile_1942} found that delinquency was not solely associated with individual-level characteristics, but that structure played a role as well. These findings led to the development of social disorganization theory which suggests that there are neighborhood-level characteristics that influence juvenile delinquency and crime rates regardless of the individual characteristics of those in the neighborhood. One of the primary mechanisms used to explain this finding is that the formation of social ties is hindered in socially disorganized neighborhoods. The systemic model, proposed by \textcite{kasarda_community_1974}, places an emphasis on the social connections between community members \parencite{bursik_jr_economic_1993, bursik_systemic_2000}. Specifically, neighborhoods with strong ties between community members can convey norms, values, and expectations of conduct for the neighborhood. The transmission of these norms and values produces a greater capacity for informal social control among community members \parencite{kornhauser_social_1978}. However, the presence of disadvantage, residential instability, and ethnic heterogeneity can hinder the development of informal social controls. Because there is an inability to develop strong informal social control mechanisms in neighborhoods marked by disadvantage, deviance and other forms of criminal activity can go unthwarted. In one of the first full tests of social disorganization theory, \textcite{sampson_community_1989} showed that there was an indirect effect of neighborhood-level structural characteristics on crime and that the strength of networks among community members mediated the relationship. Contemporary work has continued to support this finding showing the impact of structural characteristics and the role of community ties play across jurisdictions \parencite{bursik_systemic_2000, levy_triple_2020, sampson_great_2012, sampson_neighborhoods_1997}.

Another ecological theoretical framework for explaining the association between community-level characteristics and crime is the macro-level strain theory proposed by \citeauthor{agnew_general_1999} \citeyear{agnew_general_1999}. Agnew suggests that the differential distribution of crime across communities is not solely a product of weak informal social controls but also a product of the strain producing characteristics in these communities that either blocks individuals’ goals, produces a feeling of deprivation, introduces negative stimuli, or removes positive stimuli. There is also a network effect wherein individuals who interact with other highly strained individuals that become strained themselves. Communities with strain producing characteristics select strained individuals, particularly for those who are facing economic hardships. These individuals move into the deprived community and are confined to the community because of their inability to relocate. This then leads to community-level differences in crime rates because of the heightened likelihood of a criminal response to the strain experienced in the given neighborhood.

Yet, tests of the \textcite{agnew_general_1999} model are relatively limited, particularly at the neighborhood level. Of the research that has been done at the neighborhood level, findings have highlighted the role of strain and its impact on community crime rates \parencite{antunes_social_2022, warner_strain_2003}. This research indicates that strain does play a role in the production of crime rates at the neighborhood level. \textcite{warner_strain_2003} show that the role of informal social controls become statistically non-significant while community-level strain is significantly associated with violence. \textcite{antunes_social_2022} report similar findings in that community level strain, exposure to violence, and youth violence are associated even when controlling for collective efficacy. 

Importantly, the same macro-level process of social disorganization theory and macro-level strain theory that are associated with crime are also relevant for drug use and overdoses. Community-level characteristics that have been found to be associated with overdoses and overdose mortality include higher levels of poverty, ethnic heterogeneity, lower home ownership rates, and low educational attainment \parencite{chichester_pharmacies_2020, galea_income_2003, hannon_neighborhood_2006, ford_neighborhood_2017}. \textcite{ford_neighborhood_2017} show that social disorganization, a lack of social capital, and low social participation are predictive of adolescent opioid misuse. \textcite{winstanley_association_2008} also find the same relationship between social capital -- measured as participation in community based groups -- and reported opioid dependence. This could suggest that higher levels of social capital indicate greater levels of connectedness with community, family, and school. These findings indicate that stronger ties and social integration is associated with lower levels of reported opioid use. Social connectedness likely improves the ability for community members to enact informal social controls and discourage drug use. 

Additionally, \textcite{feldmeyer_community_2022} notes that structural factors, which include higher opioid prescription rates, population decline, poor community health, and the loss of manufacturing jobs are associated with higher levels of overdoses. \citeauthor{feldmeyer_community_2022}'s (\citeyear{feldmeyer_community_2022}) findings are in line with the deaths of despair argument proposed by \textcite{case_rising_2015, case_mortality_2017}. Specifically, the same ecological characteristics are associated with overdose deaths and suicides. \textcite{feldmeyer_community_2022} extend this to show the same predictors are associated with homicides as well. In a more recent article looking at drug overdoses in Passaic County, New Jersey from 2015-2019 at the block-group level, \textcite{piza_drug_2023} finds that one of the strongest predictors of drug overdoses is concentrated disadvantage. 

The empirical work investigating the association between structural variables and opioid overdoses suggest that weakened social controls (i.e., lower home ownership rates, population decline, disadvantage) and strain (e.g., loss of employment) influence opioid overdose rates. As discussed above, the formation of ties is important for informal social control. Disadvantage, population decline, and lower homeownership rates are neighborhood characteristics that hinder the formation of social connections which reduces the capacity to develop informal control mechanisms. When informal social control is present, members of a community may discourage opioid use resulting in either less use or potentially less risky use. Also, unemployment and the loss of manufacturing jobs have been found to be associated with elevated opioid overdose rates. This is particularly true for the Appalachian region of the US where many hard labor jobs have become obsolete \parencite{mclean_theres_2016}. This association suggests there is a relationship between strain and opioid overdoses, particularly for communities facing loss of employment opportunities and economic decline. Individuals may be strained because of their inability to achieve monetary goals and are unable to relocate given a variety of reasons including their economic situation. Communities facing economic hardships produce strain and retain strained individuals, which then leads to various forms of coping which could be opioid usage \parencite{monnat_factors_2018, monnat_contributions_2019}. While these findings suggest that policies should be implemented at the macro-level to address disadvantage, economic decline, and job loss to help facilitate the development of informal social control and reduce strain, there is a practical implication of this overlap between macro-level and neighborhood-level characteristics, crime, and opioid overdoses. Namely, the potential for police to be the first-responder at an opioid overdose given that they are also unevenly distributed throughout a city in areas marked by a higher volume of calls for service to crime-related incidents.

\subsection{Place and Policing}

Police work is often tied to space and time \parencite{fyfe_police_1991}. One line of research suggests that this is due to legal factors, such as calls for service, traffic incidents, or crime prevalence (see \cite{wilson_dilemmas_1968}). For instance, police patrol is often spatially tied to local crime trends and calls for service, as officers are aware of areas with higher call volumes \parencite{klinger_negotiating_1997}. Deploying police resources based on need – defined as volume of calls for service – is the underpinning of the “deployment hypothesis” (but see \cite{alexander_war_2010, beckett_penal_2010}). Police departments have attempted to maximize the utility of their limited resources by having officers focus on high-crime areas \parencite{skogan_fairness_2004, weisburd_reforming_2003}. 

In line with the deployment hypothesis, hot spots policing has gained traction as a way to reduce various offense types such as violent, property, disorder, and drug offenses \parencite{braga_hot_2019}. Hot spots policing is a catch all term that captures police presence at micro-places that experience high levels of crime. What police officers do in these crime hot spots varies (e.g., increased number of officers, mere presence, problem oriented policing). Regardless, this approach to reducing crime is based on the law of crime concentration which suggests that crime concentrates among a small percentage of street segments across a given city \parencite{braga_benefits_2017, weisburd_law_2015}. Because of this law of crime concentration and the neighborhood-level characteristics that produce this spatial fact, police resources tend to be clustered.

On the other hand, some argue that police are situated across neighborhoods based on extralegal characteristics and that their decision-making may also be influenced by extralegal factors \parencite{alexander_war_2010, avakame_did_1999, kochel_effect_2011}. This line of research tends to focus on the role of race and social class, and how those factors drive police behavior across interactions and neighborhoods \parencite{black_manners_1980}. However, crime, particularly violent crime, tends to cluster in neighborhoods characterized by socioeconomic disadvantage \parencite{peterson_divergent_2010}. Because of this spatial concentration, police are likely to be situated in these areas frequently.

Police officers being spatially concentrated suggests they may be better positioned to respond to overdoses quicker in places and for populations that may differ from other first-responders, given their availability and patrolling patterns. This is an important aspect of police work that is often overlooked when thinking about their role in responding to public health issues such as the opioid overdose crisis. It has been shown that although there is overlap between police and EMS data, there is some spatial dissimilarity pertaining to violence and drug activity \parencite{hibdon_concentration_2017, hibdon_going_2021}. This suggests that police and EMS data provide unique insights and are not redundant \parencite{hibdon_use_2024}. Likewise, police officers may be able to reach overdose incidents prior to EMS or the fire department which could be indicative of a quicker response time in these areas \parencite{pourtaher_naloxone_2022, white_leveraging_2022}. However, \textcite{klinger_negotiating_1997} suggests that neighborhood characteristics, such as higher crime rates, may cause officers to be less vigorous towards lower priority calls, potentially leading to a slower response time due to a combination of factors such as workload and cynical perceptions of the community members. Nonetheless, debates about the role of the police responding to opioid overdoses typically embody a philosophical argument or one that focuses on iatrogenic effects of the police involvement \parencite{doe-simkins_whose_2022, michaud_therapeutic_2023, van_der_meulen_thats_2021}. However, the spatial aspects of police involvement in opioid overdoses has yet to be explored thoroughly.

\subsection{Policing Opioid Overdoses}

Opioid overdoses are a police problem, but there are only a handful of studies that examine drivers of the police response. In fact, there is some research that suggests the police may be more likely to respond first to an overdose in rural areas \parencite{wood_overdose_2021}. \textcite{pourtaher_naloxone_2022} also find that police are frequently first on scene at an overdose in the state of New York. In rural areas, resources for other first-responders may be insufficient and response times may be delayed which may explain this finding. Moreover, in Tempe, Arizona, when police respond to a suspected opioid overdose, they are frequently first on scene \parencite{white_leveraging_2022}. But what contextual factors are associated with police arriving first at an opioid overdose? This question has yet to be explored, and as \textcite{lowder_twoyear_2020} notes, it is an important empirical question because it highlights when and where individuals who overdosed are at a greater risk of being arrested. It also highlights areas where police discretion can be better utilized to connect individuals to services such as substance use treatment.

At the incident-level, the 911 call description may influence who responds first. For instance, \textcite{atkins_disparities_2024} show that describing an overdose event as a ``non-overdose" (e.g., somthing vague or health related) was more prevalent when the victim was Black or a female. The authors suggest this could be due to a fear of police response and the potential for an arrest. Prior work has highlighted this fear is prevalent in Black communities \parencite{wagner_post-overdose_2019} and that the general fear of arrest can deter calling 911 \parencite{van_der_meulen_thats_2021}. Moreover, in communities where drug use, and specifically opioid use, is more prevalent, this avoidance of using certain terminology could have an influence in the aggregate as some harm reduction organizations have advocated for using other terminology than ``overdose" when calling 911 to avoid a police response \parencite{zagorski_how_2021}.

However, there is essentially no research on the predictors of police responding first to opioid overdoses outside of the urban/rural divide. It has only been discussed from a theoretical perspective. The present study seeks to address this gap in the literature. 

\section{\centering Current Study}

I am interested in examining if there is a) spatial variation in where police administer naloxone first at the scene of an opioid overdose, and b) why this variation exists. To reiterate, there is a paucity of research in this area. Despite some research pointing to an overlap of opioid overdoses and violent crime, there is little known about how this overlap may be associated with a police-first response to opioid overdoses. Moreover, there are other incident-level and neighborhood-level characteristics that may be associated with this outcome as well that have yet to be investigated. To evaluate the incident- and neighborhood-level characteristics of police administering naloxone first, I investigate two research questions:

1) Is there spatial variation, at the block group, in where police administer naloxone first at the scene of an opioid overdose? 

\begin{flushleft}
\noindent \textbf{H1}: Police administering naloxone first at an opioid overdose incident will vary spatially across block groups. 
\end{flushleft}

2) Why is there spatial variation in where police are more likely to administer naloxone across block groups? 

\begin{flushleft}
\noindent \textbf{H2}: The violent crime rate at the block group will be associated with police administering naloxone first at an opioid overdose incident.
\end{flushleft}

To answer these research questions, I use administrative data from Tempe Fire and Medical Rescue (TFMR) that contains calls for service to opioid related incidents. Importantly, there is an indicator of who administered naloxone prior to TFMR response. This allows me to identify which incidents received a police-first response and naloxone administration\footnote{In Tempe, TFMR responds to a larger call volume of opioid overdoses than TPD. Through communication with a TFMR analyst, police are on scene at an opioid overdose approximately 40\% to 50\% of the time. Of those, there is a subset where they are on scene first, according to \textcite{white_leveraging_2022}, they are on scene first frequently.}-- see below for a thorough description of the data. To answer the above research questions, I employ a spatial point pattern test to assess spatial variation in where police and TFMR are administering naloxone first. Second, I employ mixed-effects logistic regression models with day of month, month, and year fixed effects to control for unobserved time varying confounders (e.g., daily variation, seasonality, COVID-19, staffing shifts) to assess incident-level and block-group predictors of when and where police are more likely to administer naloxone first.

\section{\centering Methods}
\subsection{Setting}

Since 2020 the Tempe Police Department (TPD) has been involved in a collaborative effort with a social services organization – La Frontera EMPACT – to reduce opioid overdose fatalities in Tempe, Arizona. This project, called the Tempe First-Responder Opioid Recovery Project, is funded by the Substance Abuse and Mental Health Services Administration (SAMHSA). The project trained and outfitted Tempe officers with naloxone in January of 2020 and created a 24/7 crisis hotline which officers contacted following an overdose incident to initiate a response from a peer navigator (e.g., social outreach worker). In the hours and days following the overdose the peer navigator meets with the overdose survivor and their family/friends to provide information regarding potentially useful services (e.g., counseling, drug rehabilitation, housing information, transportation, etc.). The most recent numbers suggest Tempe officers have responded to more than 323 overdose incidents during the project. They have administered naloxone at 294 of those incidents which aided in a successful reversal of opioid overdose symptoms, and they have contacted the 24/7 crisis hotline approximately the same number of times. 

\subsection{Data}
The original data set from TFMR contained 3,257 calls for service to an opioid related incident from January 2017 through November 2023. Data on TFMR calls for service to opioid related incidents can be pulled from \href{https://data.tempe.gov/datasets/2daeeafd2741494c8294ca415e5a793e_0/explore?location=33.398962%2C-111.931850%2C11.94}{Tempe.gov}.\footnote{Through a contact at TFMR I was able to obtain other variables -- in addition to the public dataset -- such as whether the overdose was a fatal or nonfatal overdose.} I then geocoded this data set in \texttt{R} using the package \texttt{tidygeocoder} where I specified ArcGIS services as the method for geocoding.\footnote{The geocoding process provides a score that captures how closely an observation matches an address. It ranges from 0 to 100. Of the 3,257 opioid overdose calls for service, the lowest score in the dataset was 79.0\% and overall, there was a mean matching score of 98.6\%.} I then spatially clipped the incident data to a Tempe boundary shapefile (n = 3,209). 

Next, I used the \texttt{tidycensus} package in \texttt{R} where I specified the specific block group socio-demographic variables for the analysis. I used the \texttt{getacs} command to obtain 5-year estimates (2018-2022) for block groups in Tempe, Arizona -- the specific variables will be discussed below. Then, I remove observations where the block group total population estimate is 0 (n = 3,145) -- this also drops 2 block groups (n = 117). Lastly, I remove observations prior to 2020 because the TPD did not carry naloxone prior to 2020, and I drop observations in November of 2023 because the TFMR calls for service data was not a full month of reporting (n = 2,002). 

The offense data is pulled from \href{https://data.tempe.gov/datasets/1563be5b343b4f78b1163e97a9a503ad_0/explore?location=32.279019%2C-112.767075%2C7.97}{Tempe.gov}. The original dataset includes 320,991 observations. I began by limiting the dataset to the pertinent time period (Jan 2020 - Oct 2023; n = 102,681). I then specified two particular crime categories: violent crimes and drug offenses -- these specific categories will be described below. Then, I spatially clipped the offense data to the Tempe boundary shapefile (n = 101,534). 

\subsection{Dependent Variable}
There are two primary dependent variables for the present study: First, a dichotomous variable indicating whether law enforcement administered naloxone first at the scene of an opioid overdose. The final sample includes 288 observations (14.39\%) where law enforcement administered naloxone. There are 1,714 observations (85.61\%) where naloxone was either not administered at all, or naloxone was administered by someone other than law enforcement (e.g., layperson, TFMR, Unknown). The second dependent variable is a dichotomous variable indicating whether TFMR administered naloxone first (n = 667; 33.32\%) compared to all other outcomes (n = 1,335, 66.68\%; e.g., no administration, PD, layperson, etc.\footnote{As a sensitivity check I collapse this outcome variable on law enforcement and TFMR only. This results in 288 law enforcement administrations (30.16\%) and 667 TFMR administrations (69.84\%). This changes the reference category to more specifically assess variation between these two first-responders. The results do not substantively change.}

\subsection{Independent Variable}
The primary independent variable of interest is the violent crime rate per 1000 residents at the block group. This variable is captured by summing incidence of homicide, manslaughter, aggravated assault, robbery, and burglary with force, dividing by the ACS 5-year population estimate and multiplying by 1000.

I control for both incident level and block group level predictors. At the incident-level I control for the initial dispatch complaint. This variable in it's original form has 31 different entries. I code them into a categorical variable that has six values.\footnote{See Appendix A, Table 13 for how the variable was coded.} I separate them into \textit{overdose/poisoning}, \textit{health/medical related}, \textit{accidental injury}, \textit{public safety related}, \textit{mental health related} and \textit{other}. I also control for the sex of the individual (= 1 for female [male = 0]) and age of the individual (Younger than 20, Aged 30 - 39, Aged 40 - 59, Aged 60 - 99, Aged 20 - 29 [reference].\footnote{At the incident level, race is not included in the dataset. I discuss this in the limitations section.} At the block group level, I control for drug offense rate,\footnote{Comprised of drug paraphernalia, possession/sell/manufacture of narcotics, and possession/manufacture of prescription drugs offenses.} opioid overdose rate logged, and a series of spatial lags calculated using the spatial weight matrix (\textit{W}) to account for nearby block groups violent crime, drug offenses, and opioid overdose incidents.\footnote{A Morans I test indicated significant clustering for each variable.} Additionally, I control for disadvantage which is a principle components factor of three measures; (1) percentage unemployed, (2) percentage of families below the poverty line, and (3) percentage of single-parent households (Eigenvalue = 1.31). I then control for the percentage of non-Hispanic Black, non-Hispanic White, and Hispanic residents at the block group. I also control for the percent of land that is residential\footnote{This was calculated by using the \texttt{osmdata} package in \texttt{R} that allows for the pulling of specific land use features (e.g., bus stops, liquor stores, parks, residential land, etc.). Pulling this data provided a calculation for how much residential land was in Tempe, Arizona. Then, I spatially clipped this to the block group, where I then calculated the percent of residential land by dividing the land use that was residential by the total land in the block group, then multiplying by 100.} and the percent of owner occupied units. Lastly, I incorporate day of month, month, and year fixed effects to account to time-varying confounders such as day-to-day variation, seasonality, impacts of COVID-19, and broader year-to-year shifts that may capture changes in employment levels.

\subsection{Analytical Sample}
The final analytical sample includes 2,000 calls for service to opioid overdose incidents from January 2020 through October 2023, nested within 115 block groups in Tempe, Arizona.

\subsection{Analytical Plan}
\subsubsection{Spatial Point Pattern Test}

First, to descriptively assess the spatial variation in police and TFMR naloxone administrations I employ a spatial point pattern test \parencite{andresen_testing_2009}. This approach compares the pattern of two (or more) outcomes across spatial units (e.g., cell grids, blocks, block groups, police precincts, etc.). It has been applied in a variety of contexts such as crime concentration, stability in crime over time, and others \parencite{andresen_crime_2017, ha_spatial_2020, ratcliffe_detecting_2005}. Specifically related to the present study, I am interested in understanding which neighborhoods experience a higher volume of police administering naloxone first on scene at an opioid overdose compared to TFMR. Thus, I have two outcomes of interest to compare spatially. 

The spatial point pattern test works by first identifying which outcome is the base data and which is the test data. Here, the TFMR naloxone administrations are the base data and the police administrations are the test data. Then, 85\% of the test data is randomly sampled (with replacement) and undergoes Monte Carlo simulations (n = 200). Next, the percentages of points identified within each neighborhood can be ranked which is then used to remove the top 2.5\% and bottom 2.5\%. This allows for the creation of a 95\% confidence interval around the percentage of the points in each neighborhood. The base data is finally compared to the test data. If the base data falls within the confidence interval, then the two spatial point patterns are similar \parencite{andresen_testing_2009}.

The spatial point pattern analysis can be represented as such:

\[
    S = \frac{\sum_{i=1}^n S_i}{n}
\]

\noindent Where \(n\) is the number of block groups and \(S_i\) is equal to one if the point patterns are similar for block group \(i\). This then produces two \(S\) values. A global \(S\) and robust \(S\). The former refers to a similarity index for all block groups even if the test data was not observed in the spatial units. The latter, however, limits the similarity index to block groups with at least one observation from the test data. 

\subsubsection{Mixed-effects Logistic Regression Models}
To examine why there may be spatial variation in where police and TFMR are administering naloxone first, I employ mixed-effects logistic regression models to handle the nested structure of the data. Specifically, opioid overdose incidents nested within block groups. This allows me to have a random intercept for each block group, and to model incident and block group predictors simultaneously. The equation for this model is as follows:

\begin{align*}
   \text{logit}\left(P(Y_{ij} = 1)\right) = &\; \beta_0 + \beta_1 \text{Dispatch}_{ij} + \beta_2 \text{Age}_{ij} + \beta_3 \text{Sex}_{ij} \\
&\; + \beta_4 \text{VCrime}_{j} + \beta_5 \text{DOffense}_{j} + \beta_6 \text{OOverdose}_{j} \\
&\; + \beta_7 \text{Dis}_{j} + \beta_8 \text{Res}_{j} + \beta_9 \text{Owner}_{j} \\
&\; + \beta_{10} \text{White}_{j} + \beta_{11} \text{Black}_{j} + \beta_{12} \text{Hispanic}_{j} \\
&\; + \delta_1 W_1 \text{VCrime}_{j} + \delta_2 W_2 \text{DOffense}_{j} + \delta_3 W_3 \text{OOverdose}_{j} \\
&\; + \lambda_d \text{Day}_{ij} + \lambda_m \text{Month}_{ij} + \lambda_y \text{Year}_{ij} \\
&\; + u_j + \epsilon_{ij}  
\end{align*}

\noindent Where \(P(Y_{ij} = 1)\) represents the probability that police or TFMR administer naloxone first at incident \(i\) in block group \(j\). Then, \(\text{logit}(P)\) represents the log odds of police or TFMR administering first. \(\beta_0\) is the intercept and \(\beta_{1-3}\) correspond to the incident-level predictors (Initial call dispatch, age, and sex of individual). \(\beta_{4-12}\) correspond to the neighborhood-level predictors: violent crime rate, drug offense rate, opioid overdose rate, disadvantage, percent residential land, percent owner occupied units, percent white, percent Black, percent Hispanic. Three spatial wight matrices are included \(\delta_1\), \(\delta_2\), and \(\delta_3\) which correspond to spatial weights for violent crime, drug offenses, and opioid overdoses. \(\lambda_d \text{Day of month}_{ij}\), \(\lambda_m \text{Month}_{ij}\), and \(\lambda_y \text{Year}_{ij}\) represent day of month, month, and year fixed effects, respectively. Lastly, \(u_j\) represents the random intercept of block group \(j\) and residual error is captured by \(\epsilon_{ij}\).

\section{\centering Results}

Figure 1 depicts the results from the spatial point pattern analysis. Visually, it is clear that TFMR administering naloxone first is less concentrated and more prevalent in Southern Tempe (represented by the blue block groups). The standard \textit{S}-index is .387 and the robust \textit{S}-index is .336 (see table 2). To reiterate, an \textit{S}-index of 1 indicates perfect similarity and 0 perfect dissimilarity. However, \textcite{andresen_area-based_2016} notes that there is not a statistical test to determine similarity or dissimilarity. The general threshold that is used in the spatial point pattern test to identify similar data points is .80. Though, this is not a binary distinction. Both similarity indices in the present study range from .336 to .337 which suggests that there is spatial variation in where police are administering naloxone first compared to TFMR. This informs the next set of analyses because it suggests that there are potential neighborhood characteristics that are associated with this variation where police are administering naloxone first compared to TFMR.

Table 3 presents the results from the main mixed-effects logistic regression models. Police were less likely to administer naloxone first in neighborhoods with higher violent crime rates (\textit{OR} = 0.996 [.995, .998]). This suggests that for a unit increase in the violent crime rate, there was a .004\% decrease in the odds of police administering naloxone first. Although this represents a substantively small difference, the finding provides support for my second hypothesis. Additionally, police were less likely to administer naloxone first when the initial call for service was described as a health/medical issue (\textit{OR} = .694 [.511,.941]), a mental health issue (\textit{OR} = .373 [.223, .624]), and other (\textit{OR} = .643 [.445, .930] compared to an ``overdose/poisoning" description. Police officers were also less likely to administer naloxone first when the age of the individual was aged 40 - 59 (\textit{OR} = .615 [.446, .846]) and 60 - 99 (\textit{OR} = .400 [.165, .971]), compared to when the individual was aged 20 - 29. Percentage of owner occupied units is negatively associated with police administering naloxone first (\textit{OR} = .963 [.942, .985]). On the other hand, there were three neighborhood predictors that were associated with higher odds of a police naloxone administration: drug offense rate (\textit{OR} = 1.009 [1.005, 1.013]), percentage non-Hispanic White (\textit{OR} = 1.023 [1.006, 1.040]), and percentage non-Hispanic Black (\textit{OR} = 1.029 [1.014, 1.045]).

TFMR was more likely to administer naloxone first when the initial call for service was described as a mental health issue (\textit{OR} = 1.872 [1.300, 2.696]), and other (\textit{OR} = 2.764 [2.043, 3.740]) compared to an ``overdose/poisoning" description. Additionally, TFMR was more likely to administer naloxone first when the individual was aged 40 - 59 (\textit{OR} = 1.505 [1.040, 2.179]), compared to 20 - 29. Lastly, drug offense rate was associated with a lower likelihood of TFMR administering naloxone first (\textit{OR} = .996, [.993, .999]). 

Figure 2 provides a visualization of the statistically significant coefficients. Panel A includes neighborhood predictors and panel B contains coefficients for dispatch type and age of the individual. Confidence intervals that overlap for a specific predictor suggest that there is likely not a significant difference in the odds of police or TFMR administering naloxone first. For instance, Figure 2 shows that the violent crime rate, drug offense rate, percentage Black, mental health related calls for service, and other calls for service predict variation in when and where police and TFMR respond first to administer naloxone. Specifically, compared to TFMR, police officers were more likely to administer naloxone first in neighborhoods with higher drug offense rates, percentage Black residents, and were less likely to administer naloxone in neighborhoods with higher violent crime rates, when the incident was dispatched as mental health related or other compared to an overdose, and when the individual was aged 40 to 59.

Table 4 provides the results of a supplemental mixed-effects logistic regression model focused solely on police administering naloxone first compared to TFMR administering naloxone first. More specifically, the reference group is now TFMR administering naloxone first. This does reduce the sample size down to 891 observations nested within 106 block groups. The main results do not substantively change. See figure 3 for a visualization of the coefficients.

\section{\centering Discussion}
The present study uses spatial point pattern testing and mixed-effects logistic regression models to investigate spatial variation in where the police and TFMR are first to arrive and administer naloxone first at opioid overdose incidents. There is a paucity of research examining spatial variation in police and other first responders involvement in opioid overdoses, particularly for smaller geographical areas. 

The findings are important for two primary reasons. First, prior research has highlighted a concern of police criminalizing people who use opioids when they are on scene at an overdose \parencite{bohnert_policing_2011, van_der_meulen_thats_2021}. \textcite{lowder_twoyear_2020} show that arrest is more likely when police respond first to an opioid overdose compared to EMS. However, the authors call on researchers to investigate when and where a police-first response is more likely to provide further context to this finding as it would elucidate when and where the likelihood of a police arrest is more probable -- this study answers that call. The findings here also suggest that examining the likelihood of arrest following a police or EMS response may be limited due to the variation in when and where the two first responders are likely to be on scene first.

Second, there is a growing body of research that highlights the utility of augmenting police data with EMS data to have a more fully informed understanding of the distribution of public safety and public health issues across a city (see \cite{hibdon_use_2024}). The present study adds to this body of literature examining \textit{when} and \textit{where} variation in police and EMS (TFMR in Tempe) responses to opioid overdoses exists at the incident- and neighborhood-level. To my knowledge, the present study is the first to examine spatial variation in police and TFMR arriving first and administering naloxone at opioid overdoses. Theoretical and policy implications will be discussed below.

The results from the spatial point pattern analysis suggest that that there is variation in where police officers are administering naloxone first compared to TFMR. Although this test does not tell use why the variation exists, prior research that discusses variation in hot spots of drug activity for police and EMS data \parencite{hibdon_concentration_2017, hibdon_going_2021} point to the deployment of police to areas with higher levels of calls for service and crime as a reason for this variation (see \cite{engel_police_2003}). The \textit{S}-index in the present study is similar to \citeauthor{hibdon_concentration_2017}'s (\citeyear{hibdon_concentration_2017}) similarity index where they found that EMS and police calls for drug activity overlapped between 24\% and 36\% of the time, though, they were looking at street segments and intersections.

Moreover, the findings from Figure 1 suggest that TFMR is more spatially diverse in where they are administering naloxone first. TFMR has six stations spread out through Tempe, while Tempe PD has just three stations. Tempe PD's headquarters is in North Tempe near ASU's campus, with a substation East of that, and another substation in Southwest Tempe. TFMR's stations are generally situated in Central and North Tempe with one station in Southeast Tempe. Additionally, health-related calls for service may be less spatially concentrated leading TFMR to respond to areas that police generally are not responding first (e.g., Southern Tempe).

The results from the mixed-effects logistic regression models provide some context for when and where police are more likely to administer naloxone first. The violent crime rate at the neighborhood-level was associated with a lower likelihood of police administering first. Although the estimates are statistically significant, substantively, they are small odds ratios. Specifically, violent crime rate is associated with a 0.004\% decrease in the odds of police administering first. Moreover, comparing the two models suggest that TFMR is more likely to administer naloxone first in neighborhoods with higher violent crime rates. One explanation for this relationship is that in neighborhoods with higher levels of violent crime, police resources are spread thin handling serious crimes, which may affect their response time to non-crime calls, such as overdoses. This would position TFMR to the primary first-responder in these neighborhoods in some cases. 

Regarding the initial call dispatch, compared to an overdose description, police officers are less likely to administer naloxone first when it is health/medical related, mental health related, or other. This makes sense as health related calls will be of lower priority for police, especially when calls for service related to criminal activity are ``on the board." Also, ambiguous descriptions for calls for service where public safety is not clearly specified may result in TFMR responding first as well which speaks to the ``other" and ``mental health related" relationships. The findings here suggest that compared to police, TFMR is more likely to administer naloxone first when the call for service is described as mental health related or other. This aligns with the research conducted by \textcite{seim_bandage_2020} suggesting that while there is some overlap in police and EMS goals, there are specific organizational goals that create variation in responses to particular calls for service. 

Police are less likely to administer naloxone first when the individual is aged 40 - 59 and 60 - 99, compared to those aged 20 - 29. And when looking at the confidence intervals for each model, TFMR is more likely than the police to administer naloxone first when the individual is aged 40 - 59. The coefficients for age suggest that if the call for service is for someone who is older, the likelihood of a police response declines. This likely reflects, to some extent, the nature of the call for service being less likely to entail a public safety issue. 

Additionally, the drug offense rate in the neighborhood was associated with an increase in the odds of police administering naloxone first. It may be the case that neighborhoods marked by drug offenses alert police officers to these neighborhoods for potential future drug activity. Prior work indicates that police can reduce drug-related incidents through problem solving \parencite{mazerolle_street-level_2007}. In Tempe, police officers may be patrolling these neighborhoods which positions them to respond and administer naloxone quickly when there is an overdose. Moreover, TFMR is less likely than police to administer naloxone first in neighborhoods with higher drug offense rates. 

Also, the percentage of owner-occupied units in the neighborhood was associated with a decrease in the odds of police administering naloxone first. This indicates that police are more likely to administer naloxone first in neighborhoods with higher-levels of renter-occupied units or vacant units. Together, renter-occupied and vacant units indicate lower levels of informal social control and more opportunity for criminal activity and drug use \parencite{feldmeyer_community_2022, hannon_neighborhood_2006}. On the other hand, homeownership tends to indicate higher levels of informal social control and collective efficacy \parencite{sampson_neighborhoods_1997} which signals residents' willingness to intervene in devious acts. The finding indicates that police officers may be acting as a formal control mechanism in areas with lower owner occupied units which positions them to respond in a timely manner and administer naloxone.

Lastly, the percentage of non-Hispanic White and Black residents in the neighborhood was associated with an increase in the odds of police administering naloxone first. This is interesting because although national trends suggest that Black Americans have recently begun to be hit hard by the opioid overdose crisis \parencite{humphreys_responding_2022}, in Tempe, opioid overdose rates are positively correlated with percentage White and negatively correlated with percentage Black, at the neighborhood-level.\footnote{Percent white and opioid overdose rate (r = .299, \textit{p} < .01), percent Black and opioid overdose rate (r = -.199, \textit{p} < .01.} This seems to suggest that the higher rates of opioid overdoses in White neighborhoods draws a police response due to the frequency of overdoses. On the other hand, it may be the case that police officers are patrolling in or around neighborhoods with a higher proportion of Black residents which leads to them to being first on scene to administer naloxone in these neighborhoods. Prior research has highlighted the role of neighborhood racial composition and police behavior and the associated racially biased outcomes \parencite{fagan_street_2000, kochel_effect_2011}. Though, in the present study, this finding suggests police officers are positioned to respond quickly and provide life-saving care in these neighborhoods.

\subsection{Limitations}
There are some limitations of the present study that are worth mentioning. First, the data is derived from TFMR which are citizen-initiated calls for service. Not all opioid overdoses are called in and thus the present data are likely under counts of the true frequency of opioid overdoses. Second, at the incident-level there is no race variable which hinders the study's ability to assess \textit{who} police are administering naloxone to first. Third, the analysis is cross sectional and lacks the ability to speak to the temporal ordering of the relationships between the variables. Though, to account for time varying confounding I include day, month, and year fixed effects. Fourth, the identifier for police responding to the opioid overdose does not capture if police respond after TFMR. There is presumably a portion of incidents where they respond after TFMR. However, the present study only captures when police arrive prior to TFMR. This is not a limitation of the findings, however, to capture the totality of police involvement at opioid overdoses, it would be ideal to have all cases when police were on scene. Fifth, table 5 presents potential predictors of spatial variation in police and Fire/EMS administering naloxone first at an opioid overdose. The present study accounts for 9 of the 22 variables listed. While this table does not provide an exhaustive list of potential predictors, it does provide a road map for future studies investigating drivers of spatial variation in first responders administering naloxone. Lastly, the study focuses on Tempe, Arizona and likely lacks generalizability to other cities. However, future research should investigate the spatial variation in responsiveness to opioid overdoses in other jurisdictions at the neighborhood level to help inform first-responders and communities better address the opioid overdose crisis. 

\subsection{Implications}

From a theoretical perspective, the present study offers insights into the incident- and neighborhood-level characteristics that are associated with police arriving first and administering naloxone at an opioid overdose. First, the findings add to a body of research that has found that using EMS data in tandem with police data identifies some spatial dissimilarity in where their calls for service are located \parencite{hibdon_use_2024}. The present findings speak to \textit{why} this dissimilarity exists, within the context of police and TFMR responding to opioid overdoses. This should provide a basis to more thoroughly investigate why there is spatial variation between police and EMS data.

Second, \citeauthor{klinger_negotiating_1997}'s (\citeyear{klinger_negotiating_1997}) ecological theory of police may partially explain the relationship between violent crime rates and TFMR being more likely to respond first. Klinger posits that in high crime areas, police response to calls for service that are lower priority will be less \textit{vigorous}. The present study offers partial support for this claim. While an opioid overdose does require an expedient response to administer naloxone, in areas where there are higher rates of violent crime, it is a lower priority for police given their time and resource constraints handling crime related calls, thus EMS or Fire can respond. However, the extent to which this is explained by cynical views among officers in higher crime neighborhoods is unclear. There is also the possibility of interactive effects at the neighborhood-level. For example, it could be the case that the relationship between the violent crime rate or drug offense rate and police administering naloxone first is more pronounced in neighborhoods with higher percentage Black residents or higher levels of disadvantage (see \cite{donnelly_opioids_2021}. Future work should explore the possibility of interactions between neighborhood-level characteristics and incorporate qualitative methods to better understand police officers perceptions of the opioid overdose crisis and how they are influenced by neighborhood context.

Third, disadvantage was not associated with police administering naloxone first.\footnote{In a supplemental univariate model with disadvantage as the only predictor, it is associated with a higher likelihood of police administering naloxone first. However, adding the violent crime rate renders it marginally significant.} Prior work looking at response times to incidents of domestic violence found that disadvantage was associated with faster response times \parencite{lee_what_2017}. However, in the present study, the focus is on variation between two first-responders. It is likely the case that public health and public safety issues overlap at the neighborhood-level in areas of disadvantage which may not account for much variation in responses from police or EMS. 

Additionally, police officers were more likely to administer naloxone in neighborhoods with fewer owner occupied units. This finding aligns with prior work suggesting that homeownership rates are generally associated with less devious acts because there is higher levels of social control among residents \parencite{sampson_neighborhoods_1997}. Thus, in areas with lower owner occupied units, police officers are acting as a formal control mechanism \parencite{kelling_broken_1982} and are going to spatially positioned to respond quicker and administer naloxone first at the scene of an opioid overdose.

Lastly, police being more likely to administer naloxone first in neighborhoods with a higher percentage of White and Black residents is an interesting finding. First, opioid overdose rates are correlated with the percentage White residents suggests that police are responding first, in part due to the frequency of overdoses in these neighborhoods. On the other hand, percentage Black is not associated with higher opioid overdose rates, at the neighborhood-level. This might suggest that police officers are positioned disproportionately in and around neighborhoods with higher proportion of Black residents \parencite{black_manners_1980}. At the same time, however, police officers are quick to respond and administer naloxone at opioid overdoses. According to \textcite{klinger_negotiating_1997}, it would likely be the case that cynical attitudes may be developed towards marginalized communities which would result in a slower response. That is not the case here. Future theoretical work should investigate this further particularly for opioid overdoses and more broadly for police pubic health related calls for service. 

There are also some policy implications to consider. First, violent crime rates at the neighborhood-level being associated with a lower likelihood of police administering naloxone first may indicate that police officers are spread thin in these neighborhoods. If this is what is driving the relationship, collaboration between the police department and TFMR should be developed to enhance data sharing to facilitate more effective responses in neighborhoods where the police department may be resource deficient. This is particularly necessary if a police response is generally quicker than TFMR but is delayed in neighborhoods marked by higher rates of violent crime. Time is of the essence for an overdose, thus the collaborative effort will identify the specific areas throughout Tempe where police officers are less likely to respond first, which can inform TFMR on where to direct resources. This will ensure that individuals experiencing an opioid overdose are receiving a timely response in these neighborhoods. 

Another possibility is that this facilitates the creation of a specialized unit within either the police department or TFMR, or an innovative co-response team that devotes their time in areas where response times are delayed, potentially in higher violent crime areas to provide quick responses to opioid overdoses (see \cite{reuland_improving_2010, white_co-responder_2018}). 

Second, the complex mission of the police, specifically the conflict between public safety and public health roles, must be clearly delineated in policy when it comes to addressing the opioid overdose crisis \parencite{del_pozo_beyond_2021}. For instance, in neighborhoods with a higher share of Black residents and neighborhoods marked by higher rates of drug offenses, it is critical to approach the opioid overdose itself from a public health stance. This should entail prioritizing the health of the individual and connecting the individual to potentially useful services. In Tempe, this is the priority with the Tempe ORP in place. However, in other jurisdictions, arrests occur and are more frequent when police respond (see \cite{lowder_twoyear_2020}). A punitive approach -- prioritizing arrests -- may hinder relationships with community members \parencite{van_der_meulen_thats_2021}, exacerbate racially biased outcomes \parencite{kochel_effect_2011}, and be detrimental to the overdose victim by increasing their risk of a future overdose \parencite{binswanger_clinical_2016, ray_spatiotemporal_2023}. 

Additionally, the findings suggest that the police are capable first-responders who can provide timely life-saving care. This is particularly true for neighborhoods that have a higher proportion Black residents, White residents, renter/vacant units, and higher rates of drug offenses. This speaks to the importance of having officers outfitted with naloxone so they can provide quick life-saving care and save lives. 

Lastly, and broadly speaking, having efficient responses to reduce opioid overdose mortality is important. But, in addition to this, there is a need to expand naloxone distribution programs in neighborhoods so residents can effectively reverse the effects of an opioid overdose without the reliance on a timely response from police or EMS/Fire. This is particularly important in neighborhoods where responses may be delayed (e.g., high call for service neighborhoods). Naloxone is effective and empirically, it has been shown to reduce opioid overdose fatalities \parencite{mcclellan_opioid-overdose_2018, rees_little_2019}. 

\section{\centering Conclusion}

To my knowledge, the present study is the first to examine spatial variation in police and EMS/Fire responsiveness to opioid overdoses at the incident- and neighborhood-levels. The findings suggest that there is spatial variation in where police officers administer naloxone first at the scene of an opioid overdose. Specifically, police officers are more likely to administer naloxone first in neighborhoods with a higher drug offense rate and a higher percentage of non-Hispanic White and Black residents. They are also less likely to administer naloxone in neighborhoods with higher violent crime rates and a higher percentage of owner occupied units. At the incident level, police officers are less likely to administer naloxone when the call for service is initial dispatched as health/medical related, mental health related, or other, compared to an overdose. And police are less likely to administer naloxone first when the individual was aged 40 - 59 and 60 - 99, compared to those aged 20 -29. 

The findings add to a body of research that has focused on the concentration and overlap of police and EMS data. As discussed above, the findings provide important theoretical insights into why variation in police and TFMR responses to opioid overdoses exists. Moreover, the findings point to the need to further examine the spatial characteristics that are associated with police officers and other first responders arriving first and administering naloxone at opioid overdoses. This is acutely important given the harms associated with opioid overdoses across the country. Understanding the spatial characteristics that are associated with police and other first-responders responding to opioid overdoses can help identify areas that need more attention to provide timely emergency services.

If similar variation exists in other jurisdictions, it could help inform policies and programs for first-responders and community members, respectively. Specifically, understanding where responses may be delayed will lead to more effective responses to opioid overdoses. Namely, developing inter-agency communication, or co-response teams to effectively respond to neighborhoods in need. Also, the facilitation of naloxone distribution programs to community members will provide residents a way to reverse the effect of an opioid overdose.

\newpage

\begin{table}[htbp]\centering
\def\sym#1{\ifmmode^{#1}\else\(^{#1}\)\fi}
\caption{\centering Summary Statistics}
\begin{tabular}{l*{1}{cccc}}
\toprule
                &     Mean&       SD&      Min&      Max\\
\midrule
\emph{Dependent Variables}&         &         &         &         \\
PD naloxone admin&     0.14&     0.35&     0.00&     1.00\\
TFMR naloxone admin&     0.33&     0.47&     0.00&     1.00\\
\vspace{.05em} \\
\emph{Incident-level variables}&         &         &         &         \\
Overdose/Poisoning&     0.38&     0.49&     0.00&     1.00\\
Health/medical related&     0.32&     0.47&     0.00&     1.00\\
Accidental injury&     0.01&     0.12&     0.00&     1.00\\
Public safety related&     0.02&     0.15&     0.00&     1.00\\
Mental health related&     0.09&     0.29&     0.00&     1.00\\
Other           &     0.17&     0.38&     0.00&     1.00\\
Female          &     0.27&     0.45&     0.00&     1.00\\
Younger than 20 &     0.05&     0.21&     0.00&     1.00\\
Aged 20 - 29    &     0.34&     0.47&     0.00&     1.00\\
Aged 30 - 39    &     0.35&     0.48&     0.00&     1.00\\
Aged 40 - 59    &     0.21&     0.41&     0.00&     1.00\\
Aged 60 - 99    &     0.06&     0.23&     0.00&     1.00\\
\vspace{.05em} \\
\emph{Block group variables}&         &         &         &         \\
Violent crime rate (per 1000)&    70.49&   166.29&     0.00&  1152.67\\
Violent crime spatial lag&    55.37&    25.50&     7.89&   121.17\\
Drug offense rate (per 1000)&    35.47&    66.65&     0.00&   412.21\\
Drug offense spatial lag&    27.73&    16.87&     1.89&    76.83\\
Opioid OD rate logged&     3.42&     0.94&     0.35&     6.37\\
Opioid OD spatial lag&    35.12&    16.92&     5.00&    82.67\\
Disadvantage&     0.00&     1.00&    -1.94&     3.44\\
\% Residential land&    42.24&    29.03&     0.00&   100.00\\
\% Owner occupied units&    12.28&    10.82&     0.00&    56.01\\
\% White        &    50.69&    17.10&    10.05&    94.61\\
\% Black        &     7.43&     9.41&     0.00&    46.13\\
\% Hispanic     &    23.85&    13.36&     0.00&    84.45\\
\bottomrule
\multicolumn{5}{l}{\footnotesize Incident level: n = 2,002, Block group: n = 117}\\
\end{tabular}
\end{table}


\newpage 

\begin{table}[htbp] \centering
\def\sym#1{\ifmmode^{#1}\else\(^{#1}\)\fi}
  \caption{S-index} % Title of the table
  \begin{adjustbox}{max width=\linewidth}\begin{tabular}{l*{2}{c}}
    \toprule
     & Standard S-index & Robust S-index \\ 
    \midrule
    PD naloxone first - TFMR naloxone first & .387 & .336\\ 
    \bottomrule
  \end{tabular} \end{adjustbox}
\end{table}

\begin{figure}
    \caption{Spatial Point Pattern Test: PD and TFMR, First to Administer Naloxone Across Block Groups}
    \centering
    \includegraphics{figures/sppt-naloxone.pdf}
\end{figure}

\newpage

\begin{table}[htbp]\centering
\def\sym#1{\ifmmode^{#1}\else\(^{#1}\)\fi}
\caption{Mixed-effects Logistic Regression Models Predicting PD and TFMR first to Administer Naloxone}
\begin{adjustbox}{max width=\linewidth}\begin{tabular}{l*{2}{c}}
\toprule
                &\multicolumn{1}{c}{PD naloxone first}&\multicolumn{1}{c}{TFMR naloxone first}\\
\midrule
\emph{Independent variable}&                 &                 \\
Violent crime rate (per 1000)&0.997\sym{**} (0.001)        &1.001\sym{*} (0.001)        \\
\vspace{.05em} \\
Initial dispatch complaint (ref = Overdose/poisioning)&                 &                 \\
Health/medical related&0.684\sym{*} (0.104)        &1.326\sym{*} (0.182)        \\
Accidental injury&0.477 (0.502)        &0.770 (0.318)        \\
Public safety related&0.273 (0.282)        &0.519 (0.322)        \\
Mental health related&0.391\sym{**} (0.096)        &1.998\sym{**} (0.365)        \\
Other           &0.649\sym{**} (0.108)        &2.948\sym{**} (0.473)        \\
\vspace{.05em} \\
Female          &0.774 (0.127)        &1.126 (0.123)        \\
\vspace{.05em} \\
\emph{Block group predictors}&                 &                 \\
Violent crime spatial lag&1.008 (0.006)        &0.997 (0.004)        \\
Drug offense rate (per 1000)&1.007\sym{**} (0.002)        &0.994\sym{**} (0.002)        \\
Drug offense spatial lag&0.995 (0.007)        &1.005 (0.006)        \\
Opioid OD rate logged&0.937 (0.096)        &1.029 (0.087)        \\
Opioid OD spatial lag&0.996 (0.007)        &0.999 (0.006)        \\
\% Unemployed   &0.993 (0.012)        &0.999 (0.005)        \\
\% Residential land&1.001 (0.003)        &1.003 (0.002)        \\
\% Owner occupied units&0.978\sym{*} (0.009)        &0.995 (0.006)        \\
\% White        &1.012 (0.008)        &0.997 (0.005)        \\
\% Black        &1.025\sym{**} (0.007)        &0.987 (0.007)        \\
\% Hispanic     &1.007 (0.008)        &0.996 (0.006)        \\
Constant        &0.098\sym{**} (0.074)        &0.683 (0.355)        \\
\midrule
Random intercept&1.000\sym{**} (0.000)        &1.000 (0.000)        \\
\midrule
Observations    &     1,994        &     1,994        \\
Block groups    &  115        &  115        \\
AIC             & 1645.796        & 2464.457        \\
BIC             & 1830.527        & 2649.188        \\
\bottomrule
\multicolumn{3}{p{16cm}}{\footnotesize Exponentiated coefficients displayed. Clustered standard errors in parentheses. Month and year fixed effects included but not shown. Mean VIF value = 2.46, single highest VIF value = 7.49.}\\
\multicolumn{3}{l}{\footnotesize \sym{*} \(p<0.05\), \sym{**} \(p<0.01\), \sym{**} \(p<0.001\)}\\
\end{tabular} \end{adjustbox}
\end{table}


\newpage

\begin{table}[htbp]\centering
\def\sym#1{\ifmmode^{#1}\else\(^{#1}\)\fi}
\caption{Mixed-effects Logistic Regression Models Predicting PD and TFMR First to Administer Naloxone: PD and TFMR only}
\begin{adjustbox}{max width=\linewidth}\begin{tabular}{l*{2}{D{.}{.}{-1}}}
\toprule
                &\multicolumn{2}{c}{Police naloxone admin}\\\cmidrule(lr){2-3}
                &\multicolumn{1}{c}{(1)}        &\multicolumn{1}{c}{(2)}        \\
\midrule
\emph{Independent variable}&                 &                 \\
Violent crime rate (per 1000)&                 &0.995\sym{**} (0.001)        \\
\vspace{.05em} \\
Initial dispatch complaint (ref = Overdose/poisioning)&                 &                 \\
Health/medical related&                 &0.659\sym{*} (0.135)        \\
Accidental injury&                 &0.932 (1.005)        \\
Public safety related&                 &0.280 (0.193)        \\
Mental health related&                 &0.243\sym{**} (0.069)        \\
Other           &                 &0.347\sym{**} (0.072)        \\
\vspace{.05em} \\
Female          &                 &0.779 (0.149)        \\
\vspace{.05em} \\
Age range (ref = Aged 20 - 29)&                 &                 \\
Younger than 20 &                 &0.786 (0.279)        \\
Aged 30 - 39    &                 &0.881 (0.200)        \\
Aged 40 - 59    &                 &0.454\sym{**} (0.100)        \\
Aged 60 - 99    &                 &0.272\sym{**} (0.116)        \\
\vspace{.05em} \\
\emph{Block group predictors}&                 &                 \\
Violent crime spatial lag&                 &1.011 (0.007)        \\
Drug offense rate (per 1000)&                 &1.013\sym{**} (0.002)        \\
Drug offense spatial lag&                 &0.985\sym{*} (0.008)        \\
Opioid OD rate logged&                 &0.994 (0.149)        \\
Opioid OD spatial lag&                 &1.000 (0.001)        \\
Disadvantage&                 &1.079 (0.104)        \\
\% Residential land&                 &0.995 (0.003)        \\
\% Owner occupied units&                 &0.978 (0.013)        \\
\% White        &                 &1.024\sym{*} (0.011)        \\
\% Black        &                 &1.036\sym{**} (0.012)        \\
\% Hispanic     &                 &1.013 (0.010)        \\
Constant        &0.409\sym{**} (0.036)        &0.234 (0.263)        \\
\midrule
Random intercept&1.073 (0.067)        &1.000 (0.000)        \\
\midrule
Observations    &      955        &      891        \\
Block groups    &  112            &  106        \\
AIC             & 1170.921        & 1060.195        \\
BIC             & 1180.644        & 1381.282        \\
\bottomrule
\multicolumn{3}{p{20cm}}{\footnotesize Exponentiated coefficients; Clustered standard errors in parentheses; Day of month, month, and year fixed effects included but not shown; Mean VIF value = 2.30, single highest VIF value = 9.95. Reference group is now TFMR administering first -- Not all other observations.}\\
\multicolumn{3}{l}{\footnotesize \sym{*} \(p<0.05\), \sym{**} \(p<0.01\), \sym{**} \(p<0.001\)}\\
\end{tabular} \end{adjustbox}
\end{table}


\newpage

\begin{figure}
    \caption{Coefficient Plot Predicting First to Administer Naloxone}
    \centering
    \includegraphics{figures/me-logit-coef-comb.pdf}
\end{figure} 


\newpage
\begin{figure}
    \caption{Coefficient Plot Predicting First to Administer Naloxone: PD and TFMR only}
    \centering
    \includegraphics{figures/me-logit-coef-comb-sens.pdf}
\end{figure}

\newpage
\begin{table}[htbp] 
    \centering
    \def\sym#1{\ifmmode^{#1}\else\(^{#1}\)\fi}
    \caption{Modeling Spatial Variation in Naloxone Administration} % Title of the table
    \begin{adjustbox}{max width=\linewidth}
        \begin{tabular}{lc}
            \toprule
            \textbf{Potential Predictors} & \textbf{Predictors Modeled} \\ 
            \midrule
            \textbf{Systems-level}$^{a}$ & \\
            CAD system filtration of calls for service & \(\times\) \\
            \midrule
            \textbf{Community-level} & \\ 
            Violent crime rate & \checkmark \\ 
            Drug offense rate & \checkmark \\ 
            Opioid overdose rate & \checkmark \\ 
            Socioeconomic status & \checkmark \\ 
            Racial demographics & \checkmark \\
            Residential land use & \checkmark \\
            Total call volume: Police & \(\times\) \\
            Total call volume: Fire/EMS & \(\times\) \\
            Naloxone availability & \(\times\) \\ 
            Drug market shifts & \(\times\) \\
            \midrule
            \textbf{Agency-level}$^{b}$ & \\
            Staffing levels & \(\times\) \\
            Variation in deployment of resources & \(\times\) \\
            \midrule
            \textbf{Incident-level} & \\
            Initial call for service description & \checkmark \\
            Age of victim & \checkmark \\
            Sex of victim & \checkmark \\
            Location of incident (e.g., street, apartment, etc.) & \(\times\) \\
            Response time & \(\times\) \\
            Race of victim & \(\times\) \\
            Time of day & \(\times\) \\
            \midrule
            \textbf{Individual-level} & \\
            Experience and/or training to recognize an opioid overdose & \(\times\) \\
            Personal biases towards drug use/naloxone & \(\times\) \\
            \bottomrule
            \multicolumn{2}{p{17cm}}{\footnotesize $^{a}$Dispatch systems vary in how PD and EMS/Fire are notified of calls for service (e.g., one system or two), this could drive some spatial variation; $^{b}$Agency-level predictors could be accounted for in the day-, month-, and year-fixed effects}\\
        \end{tabular} 
    \end{adjustbox}
\end{table}