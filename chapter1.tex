\chapter{The Opioid Crisis and The Role of the Police}

% outline:
% discuss the opioid crisis - trends, drivers (theory, geography), costs, and responses (state, and local policy) lead to police being situated to respond
% BUT, there are debates about police involvement - criminalization (warrant checking, arrests), iatrogenic (drug seizures, naloxone stacking), fear of calling police among PWUOs, and broadly "police are law enforcers" (need cites)
% counter narrative re police - have been involved in public health issues since their inception (not just law enforcers); akin to a trained social worker; interact with vulnerable pops/PWUDs frequently; communities need professional and political positions to provide services that evolve over time -- thus, experimentation is good to allow for progression (Mill, see monaghan)
% police have adapted their methods frwquently over time. They have addressed crime in different ways (HSP, POP, Proactive, etc), adapted with respect to homelessness (co-response cites), mental health issues (co-response issues), low level offenses (LEAD, diversion). These adaptations are about progress, experimentation. They are necessary to effectievly address community concerns, and varying political adn social contexts. Thus, adopting naloxone fits within this framework of experimentation to align with community needs. Otherwise, community needs are unmet or are unmet in inefficent or ineffective ways. 

\section{Introduction}

The last two decades have been marked by a growing concern regarding opioid overdose trends, their impact on society, and the dearth of effective responses to combat this public health crisis. Since the 1990s, opioid overdose fatalities in the U.S. have increased exponentially. Opioid overdose fatalities reached an all-time high in 2021. Provisional estimates suggest a slight increase from 2021 to 2022 as well (82,998). According to the CDC, there were 80,411 fatal opioid overdoses in 2021 which eclipsed the previous high of 68,630 in 2020 \parencite{national_institute_on_drug_abuse_drug_2023}. These trends have been costly both socially and economically. In 2017, the overall monetary cost of the opioid overdose crisis was estimated to be \$1 trillion \parencite{luo_state-level_2021}. Given that opioid overdoses have reached unprecedented highs in recent years, this figure is likely to have increased. This public health crisis has led the Biden administration to award over \$1.6 billion in grants for investments into programmatic responses to address opioid overdoses in the U.S \parencite{us_department_of_health_and_human_services_biden-harris_2022}.

Overdose trends have forced agencies and policymakers to consider approaches to effectively combat the ongoing crisis. Many of the approaches taken include various state laws (e.g., Good Samaritan Laws), prescription drug monitoring programs (PDMPs), naloxone access laws (NALs), insurance strategies, and others \parencite{haegerich_evidence_2019}. One such approach is collaboration between public safety and public health entities. Given that the police mission entails a wide variety of tasks, they frequently interact with people who use drugs (PWUDs), and respond to overdoses, the police are well-positioned to be a part of a response to opioid overdoses, including the provision of life-saving care such as the administration of naloxone. The police can then act as a conduit to services through referral or diversion mechanisms allowing public health agencies to provide information and help following contact with police.  Police may also facilitate a role within the context of post-overdose outreach with a social worker or harm reduction specialist \parencite{bagley_scoping_2019}.

An important empirical finding that must be acknowledged in developing responses to the opioid overdose crisis is the spatial distribution of these events and the structural factors that are associated with overdoses. Neighborhood-level structural characteristics such as concentrated disadvantage, income inequality, job loss, and population decline have been found to be associated with overdose mortality \parencite{carter_spatial_2019, feldmeyer_community_2022, ford_neighborhood_2017, piza_drug_2023}. Likewise, these neighborhood-level characteristics, particularly concentrated disadvantage, is also associated with violent crime rates \parencite{peterson_divergent_2010}. The police are often operating within these areas of structural disadvantage due to the disproportionate spatial distribution of crime. A byproduct of this spatial concentration makes police work ecologically different from other first-responders. Not only do they outnumber other first-responders \parencite{lurigio_opioid_2018} but spatially, they are patrolling and often in areas where opioid overdoses are disproportionately concentrated. This aspect of police work may provide a practical foundation for officers to frequently interact with PWUDs and administer naloxone to those overdosing in a way that other first-responders are unable to.

However, outfitting officers and other first-responders is not enough to combat the opioid overdose crisis as this approach is a short-term solution to reduce harm \parencite{goodison_law_2019}. Some jurisdictions have taken it a step further to incorporate social service organizations to become involved following police contact. Officers on scene at an overdose may act as a bridge to services either through referral or diversion in lieu of arrest \parencite{collins_seattles_2017, paul_meeting_2018}. Other programs may involve a post-overdose outreach team that combines both public safety and public health agencies to facilitate a response to those who previously overdosed. The social service organizations are then able to discuss with the individual who overdosed the services that are available and potentially helpful. 

The research on collaborative approaches in addressing the opioid overdose crisis is lacking given it is a relatively new approach to addressing opioid overdoses. However, these approaches have been described as a promising path towards addressing the crisis \parencite{haegerich_evidence_2019, yatsco_alternatives_2020}. Post overdose outreach programs have been found to reduce fatal overdoses \parencite{xuan_association_2023}. Likewise, collaborative post-overdose outreach between police and public health entities suggest that the police can play an important role in reducing opioid overdoses \parencite{donnelly_law_2022}. The police mission, their proximity to the issue, and authority positions them to respond to opioid overdoses. Their ability to partner with public health organizations to reduce opioid overdose fatalities may provide a promising a path. Given the opioid overdose crisis has reached historic highs in recent years, it’s a pressing issue to develop effective responses. 

\section{Overview of the Opioid Overdose Crisis}
In 2021, there were 106,699 drug overdose fatalities which accounted for more deaths than motor vehicle fatalities (45,404) and all firearm deaths combined (48,830) \parencite{center_for_disease_control_and_prevention_injuries_2023}. Drug overdose is one of the leading causes of accidental death in the U.S., and for those aged 1 to 44, it is the leading cause of accidental death. Dying from a drug overdose used to be relatively rare \parencite{center_for_disease_control_and_prevention_national_2023}. The drug overdose death rate hovered around 2 per 100,000 in the 1980s and early 1990s \parencite{warner_drug_2011}. By 2000, it was over 5 per 100,000 and a decade later it was over 10 per 100,000 \parencite{rudd_increases_2016}. In 2020, it roughly tripled to just over 28 per 100,000 \parencite{hedegaard_drug_2021}. More recent estimates from 2022, put the drug overdose death rate at 32.6 per 100,000 \parencite{spencer_drug_2024}. This increase in drug overdoses has been driven by the use of opioids, and as of late, synthetic opioids (e.g., fentanyl and its analogs). The exponential growth of fatal opioid overdoses has been descriptively categorized into three notable waves. The first wave began in the 1990s, driven primarily by supply side mechanisms related to the over prescription of painkillers \parencite{kolodny_prescription_2015}. Other demand-side factors contributed to the uptick in opioid usage as well such as economic despair, aging populations, etc. \parencite{dasgupta_opioid_2018}. The second wave of opioid overdoses was a shift away from prescription medications to heroin as the former became increasingly regulated. Younger individuals’ use of opioids declined while heroin use increased, suggesting a form of dependency and a transition into a new market \parencite{unick_us_2017}.The third wave is marked by the introduction of synthetic opioids (i.e., fentanyl) into the drug markets around 2014. Fentanyl’s emergence into the market in 2014 is suggestive of supply-side innovation, namely because fentanyl is cheaper than heroin. There wasn't much demand for fentanyl early on, although the demand for it grew over time \parencite{powell_evolving_2021}. Scholars who have conducted ethnographic work have found that users vary in their desire to use fentanyl \parencite{carroll_exposure_2017, mars_illicit_2019}. Additionally, because of its potency \parencite{ciccarone_fentanyl_2017} and its prevalence in other substances – such as methamphetamine, cocaine, and benzodiazepines – fentanyl has drastically impacted the mortality rates associated with overdoses for a variety of drug types, particularly cocaine \parencite{humphreys_responding_2022}.

The cost of a fatal overdose is estimated to be \$11.5 million (e.g., loss of life, health care, loss of economic productivity), while the estimate for a non-fatal overdose is \$2,651 (e.g., naloxone and hospital costs) \parencite{center_for_disease_control_and_prevention_economic_2020}. Using recent estimates from the CDC, there were 80,411 drug overdose fatalities that involved an opioid in 2022. Based on back-of-the-envelope calculations, the cost of 81,806 opioid overdose fatalities is roughly \$940.76 billion, an approximate 58\% increase from 2017 estimates (\$549.69 billion). This increase in cost highlights the immediate need to evaluate programs and policies devised to combat OUD and reduce opioid overdose mortality. 

\section{The Role of the Police}
The police are constantly called upon to handle a host of societal problems. Quite frequently, the problems that police deal with are not crime related \parencite{mastrofski_police_1983, wilson_dilemmas_1968}. Indeed, the institution of policing can be thought of as a last resort entity that handles societal issues that seem to permeate through other institutional failures \parencite{bittner_police_1967}. It follows that the role of police in society is not simply to control crime or apply the law in a strict manner. Order maintenance, for instance, is a goal of the police that entails addressing public disorder, which may require the application of the law, yet often times can be resolved informally \parencite{kelling_broken_1982}. Although, the theoretical framework of broken window's policing was used as the foundation of a controversial enforcement strategy using stop, question, and frisk \parencite{white_federal_2016}. Nonetheless, the order maintenance role of the police, at least as described by \textcite{kelling_broken_1982}, does not emphasize the use of aggressive policing tactics to ensure public safety or to quell public disorder. Within this framework, police officers’ day-to-day activities are more reminiscent of a social worker or a street corner politician \parencite{muir_police_1979}. In a similar vein, \textcite{del_pozo_arrest_2022} discusses the role of the police in responding to issues of homelessness and describes a foundational goal of the police:

\begin{quote}
...the police should also broker and enforce the fair terms of social cooperation in public spaces when people lay legitimate but competing claims to them. In these cases, police need to resolve conflicting rights claims; if we let people sort it out for themselves, the results can often be illiberal and counter to a commitment to democratic pluralism (pg. 2).
\end{quote}

\noindent To that end, the police are not solely law enforcers but are arbiters of the use of public space. And community members all have a claim to use public space, however, in some instances, the behaviors engaged in will draw a police response because there are competing claims over the use of public space \parencite{del_pozo_arrest_2022}. Importantly, the behaviors need not be unlawful for competing claims to arise or for the police to respond \parencite{thacher_order_2014}. It is the public nature of the conflict at hand that may necessitate a police response.

Yet, there are arguments that the police mandate must be constrained. This sentiment is echoed by both police critics and the police themselves in some cases \parencite{hodges_booker_2_2017}. Some argue that the police should be abolished \parencite{vitale_end_2017}, others argue that their function must be restricted to more specific domains. Nonetheless, when the police are responding to and attempting to resolve issues in the community that range from mundane disputes between neighbors to murder, it is evident that there is a wide range of societal problems that fall into the police domain. \textcite{bittner_functions_1970} famously stated that "no human problem exists, or is imaginable, about which it could be said with finality that this certainly could not become the proper business of the police” (p. 244). This quote not only highlights the broad complexity of the police mission, which has been described as “the impossible mandate” \parencite{manning_police_1978}, but it captures the fact that the police respond to and attempt to resolve a host of community problems which includes public health issues.

Frequently, the police are called upon by citizens to resolve whatever issue may be at hand. This could involve a violation of the law – which is precisely the job of the police. However, in some instances, these issues do not necessitate the enforcement of the law by an officer and may indeed entail more of a public health response, or at a minimum, a non-criminal justice response. A practical reason the police respond to and handle public health issues is that their role in society is much broader than the enforcement of the law and their day-to-day activities are often intertwined with public health problems \parencite{wood_private_2020}. Additionally, calls for service to crime-related events may turn out to be health related \parencite{ratcliffe_policing_2021}. A more philosophical reason the police handle public health issues is that if the police were to be described as having a single mission, it would be to protect the citizenry as they are called upon to resolve human problems that may require the use of force \parencite{bittner_functions_1970, skolnick_above_1993}. Responding to public health issues falls into the purview of the police based on Bittner’s prominent depiction of the role of the police in society.

Additionally, the police provide time sensitive life-saving measures to those in need. Examples include someone who is in a burning car, or house; someone who is drowning; animals that need rescuing; debris that has someone trapped. It's fair to say that these examples are not a part of the typical schema that one has when they envision police duties. It is also fair to say most individuals do not have a problem when a police officer provides life-saving measures in any of the instances described above. An opioid overdose is a time sensitive incident where life-saving measures are required. Why is an opioid overdose viewed differently? Below I discuss the issues surrounding police involvement in substance use and opioid use specifically. I then return to this question discussing some of the reasons police ought to respond to opioid overdoses. 
 
\subsection{Police Involvement in Substance Use}

Police involvement in drug-related issues has been contentious as U.S. law enforcement has historically relied on arrests and incarceration to reduce harm and create safe communities. The focus on arrests and incarceration embodies supply-side strategies which include drug busts, undercover operations, and an emphasis on drug task forces. This approach was the product of political maneuvering to create a War on Drugs which began in 1973 under President Nixon and was reified under President Reagan in 1982 \parencite{cooper_war_2015, lynch_theorizing_2012}.

To further emphasize the police role with respect to drug use, both federal- and state-level legislation supported the notion that drug use was a public safety issue with harsh penalties for those in possession of or selling narcotics. Examples include the New York Rockefeller Drug Laws \parencite{drucker_population_2002} and the disparate sentencing for crack cocaine versus cocaine \parencite{shein_racial_1993}. Policies were implemented that incentivized police to arrest those suspected of drug possession or selling with the passage of the 1984 Omnibus Crime Bill. This allowed police to confiscate any assets believed to have been purchased with money obtained via the drug market \parencite{beckett_politics_2003}. Drug arrests increased through the 1980s and 1990s with a disparate impact on minority communities \parencite{western_punishment_2006}. The enforcement focused approach that the police have occupied when addressing drug-related issues has been more costly than beneficial \parencite{baum_smoke_1996}.

However, the opioid overdose crisis has a different climate surrounding it than prior drug crises. \textcite{fedders_opioid_2019} describes the shift in police work surrounding the opioid overdose crisis as "opioid policing" while others have referred to the new role as "harm-reduction policing" \parencite{beckett_uses_2016}. The emphasis on treatment and harm-reduction has been the focus of policymakers for the last two decades during the growing opioid overdose crisis. The purported reasons for the shift from enforcement to public health focused vary. Some argue that this is a result of the stereotypical opioid user being White and from suburban/rural areas as opposed to Black, urban dwellers \parencite{hart_opioid_2019}. Although, the demographics of those fatally overdosing from opioids has shifted in recent years \parencite{humphreys_responding_2022}. Further explaining the shift in approaches, others point to the fact that the opioid overdose crisis is a product of greedy pharmaceutical companies who aggressively marketed Oxycontin to physicians and provided illegal payments to physicians in exchange for writing Oxycontin prescriptions \parencite{hoffman_purdue_2020}. Moreover, it was the prescription of painkillers that played a large role early in the crisis \parencite{dasgupta_opioid_2018}. The latter two explanations arguably remove some culpability away from the user, which may be associated with a move towards treatment. And more broadly, the War on Drugs of the 1980s and 1990s is seen not only as costly, but a failure. Thus, the enforcement heavy approach to drug related issues was altered to include a public health model. 

The cultural, political, and programmatic shift to focus on treatment and harm-reduction has influenced policy at both the federal- and state-level. The shift towards a public health model for police has primarily entailed diversion and referral programs to get PWUDs in contact with social services. However, these changes can be hampered when police are involved. Officer discretion can prevent policy change from being enacted at the street-level \parencite{friedman_intersectional_2021}. Police officers are a mediator between the law on the books and the law in practice – they are the brokers of the rules of social cooperation \parencite{del_pozo_arrest_2022}. As an example, Good Samaritan Laws provide immunity for the individual who overdosed and others on scene who call for help and administer aid. However, officers have the discretion to arrest for outstanding warrants, and other offenses at the scene of the overdose. To address this issue, some have argued for a shift in the goal of the police, particularly as it applies to the policing of drugs. \textcite{del_pozo_beyond_2021} argue for public health ethics within police departments which would translate to a holistic approach focused on the health and general well-being of citizens. Survey research has indicated that some officers see the value in adopting harm reduction methods that focus on the health of citizens, particularly as it pertains to the carrying of naloxone and collaboration with outreach organizations \parencite{banta-green_police_2013, lloyd_its_2023, white_moving_2021}. And body-worn camera footage has shown that officers infrequently arrest overdose survivors and can act as a conduit to service providers \parencite{white_leveraging_2022}. However, concerns surrounding police involvement still linger. 

\subsection{Debating the role of the police in the opioid crisis} 

Now I return to the question of "Why is an opioid overdose different?" Below, I will discuss some of the reasons why an opioid overdose is different than the other emergency situations described above. Broadly, the primary issue of police involvement in opioid overdoses is the concept of iatrogenesis. Police officers who respond to opioid overdoses and intend on helping, can end up causing harm. For instance, officers may arrive on scene at an opioid overdose and check for warrants and arrest individuals on scene, including the overdose survivor. This is viewed as good police work, to some. It's not problematic in and of itself that officers check for warrants and make arrests. After all, this is a part of their job. It's problematic because it is within the context of the opioid crisis it is believed by some that solving issues related to drug use, and more specifically opioid use, is contingent upon police arresting overdose survivors and/or people on scene at the overdose. Prior research has found that the likelihood of arrest given a police response is higher compared to receiving an emergency medical services (EMS) response \parencite{lowder_twoyear_2020}. This finding, in addition to the history of police involvement in drug issues, has resulted in those on scene at an overdose reporting that they are fearful of a police response. This fear corresponds to lower likelihood of calling 911 \parencite{bohnert_policing_2011, van_der_meulen_thats_2021}. Moreover, the fear of a police response could be driving a difference in terminology on the part of the 911 caller in an attempt to avoid receiving a police response when the individual overdosing is Black \parencite{atkins_disparities_2024}. The authors note that 911 callers were more likely to describe the emergency as a medical issue, as opposed to an overdose when the individual was Black. 

Additionally, PWUOs will likely not benefit from incarceration. In fact, PWUOs’ likelihood of fatal overdose is heightened following incarceration due to their lowered opioid tolerance levels \parencite{binswanger_clinical_2016, merrall_meta-analysis_2010}. This concern has been replicated within the context of policing focusing on drug seizures in a recent study by \textcite{ray_spatiotemporal_2023}. The authors find that following opioid-related seizures, non-fatal overdoses, fatal overdoses, and naloxone administrations increased at 7, 14, and 21 days within a distance of 100, 250, and 500 meters, respectively. The authors note the theoretical underpinnings are akin to the logic behind the post-release incarceration and increased fatal overdose likelihood \parencite{seaman_mortality_1998}. Mainly, following a drug bust, users may seek out new dealers which comes with risks, particularly uncertainty regarding the safety of the dose from a new dealer and unknown tolerance levels \parencite{ray_spatiotemporal_2023}.

The research from above highlights the potential iatrogenic effects of police involvement in opioid related issues. The tension here points to what \textcite{carroll_police_2023} describe as the "police paradox." Which largely refers to the conflicting goals between public health and public safety agencies. Specifically, law enforcement agencies do exactly that, they enforce the law and within the context of opioid overdoses, they may rely on arrest to achieve their goals of public safety. However, public health agencies tend to focus on limiting criminal justice contact and creating pathways to treatment and other resources. Despite the fact that harm reduction policing has gained traction over the last decade, this concern still looms large.

Because of the tension between public safety and public health goals with respect to drug use, opioid overdoses are not the same as someone trapped in a burning vehicle or house. These are qualitatively different situations which needs to be acknowledged. They are similar, however, in that there is someone who needs immediate aid. Indeed, just like a burning house, an opioid overdose requires a quick response. The risk of a delayed response to an opioid overdose is not just death, which is of course the worst potential outcome. A delayed response can allow for respiratory depression to set in which has the potential to cause cerebral hypoxia \parencite{winstanley_neurocognitive_2021}. This can lead to later brain damage, cognitive impairments, and organ damage. If police officers are able to respond to overdoses quicker than other first responders, as other research has suggested is the case \parencite{pourtaher_naloxone_2022, white_leveraging_2022}, they can play a vital role in reducing harm. Moreover, assuming that in a given jurisdiction police officers are responding quicker than other first responders to opioid overdoses, and 911 callers are avoiding labeling an emergency as an overdose and instead describing it as a medical issue \parencite{atkins_disparities_2024}, this could create disparities in overdose mortality between Black and White citizens. 

Also, there is a more philosophical reason why the police ought carry naloxone and be a part of the response to opioid overdoses. \textcite{monaghan_broken_2022} provides a compelling argument on the basis of \textcite{mill_liberty_1998}'s  "experiment in living." 

\begin{quote}
As it is useful that while mankind are imperfect there should be different opinions, so is it that there should be different experiments of living; that free scope should be given to varieties of character, short of injury to others; and that the worth of different modes of life should be proved practically, when anyone thinks fit to try them (pg. 193).
\end{quote}

Monaghan analogizes this with the idea of "experiments in working" and effectively argues that in order to progress in an ever evolving political and social context, police departments must adapt or experiment with new approaches to address community problems. This provides a sound framework for why the police ought to carry and administer naloxone. The police have adapted and experimented throughout history. The automobile and two-way radio drastically altered the way police conducted their business. Likewise, policing strategies have become more evidence-based with the implementation of hot spots policing and problem oriented policing \parencite{braga_hot_2019, hinkle_problemoriented_2020}. Twenty-first century technology has also influenced policing with the diffusion of body-worn cameras (BWCs), for example \parencite{white_cops_2020}. These changes represent experiments in working, as \textcite{monaghan_broken_2022} described it. They represent attempts to be more efficient and responsive to the community. Community needs and concerns shift, the opioid overdose crisis represents a community problem that has become increasingly worse over time. Given the police are frequently at the scene of an overdose, and in many cases the first one scene, it is imperative they can provide meaningful aid (i.e., administering naloxone) to someone who is experiencing an overdose. 

\subsection{Beyond Naloxone}

The argument made above regarding police officers carrying naloxone can be extended to police collaborating with public health agencies to provide more accessible resources to PWUOs. Not only have police officers' supported carrying and administering naloxone, they have also shifted to support innovative and collaborative ways to combat rising opioid overdoses \parencite{pourtaher_naloxone_2022, white_moving_2021}. Traditional methods of addressing opioid use have been ineffective and there is an opportunity to experiment with new ways of approaching the opioid overdose crisis. 

With police officers frequently interacting with PWUOs, being on scene at overdoses, and having information on people PWUOs and have had criminal justice contact, it makes for a seamless transition to then refer overdose survivors to services. Moreover, if contacting overdose survivors through traditional means is inadequate or ineffective, involving the police may provide a necessary bridge that helps improve engagement rates. Being on scene at an overdose and/or responding to the hospital once the overdose victim is transported, allows the officer to refer the individual to services and provide necessary information to social service workers so they can contact the overdose survivor. Thus, the collaboration between law enforcement and public health agencies has the potential to connect with more overdose survivors through increased communication. 

\section{Dissertation Studies}
It is the debate surrounding police involvement in the opioid crisis that informs this dissertation. This dissertation will investigate aspects of the Tempe First-Responder Opioid Project which will provide important findings to inform policy surrounding police involvement in opioid-related issues.

The first two studies of this dissertation investigate important antecedents that may contribute to the effectiveness of a collaborative response. First, I will provide descriptive statistics and visualizations of the spatial distribution of crime and opioid overdoses as well as TPD and TFMR naloxone administrations in Tempe, Arizona. Then, I will employ negative binomial regression models to assess the neighborhood-level predictors of TPD and TFMR naloxone administrations across block-groups. Examining opioid overdoses and naloxone administrations spatially provides insight into key aspects of the police being involved in opioid overdose incidents. Namely, the overlap of disadvantage, crime, and opioid overdoses provide a practical justification for the police to be involved in responding to overdoses, providing life-saving care, and administering naloxone given they are frequently in areas marked by high crime. Prior work has shown that EMS and police call for service data can differ spatially \parencite{hibdon_concentration_2017, hibdon_going_2021}. This chapter will examine the block-group level socio-demographic factors that are associated with spatial variation in TPD and TFMR naloxone administrations. Outfitting police officers with naloxone and having them be a primary mechanism for the connection to social services (e.g., through referral, diversion, or co-response models) may allow the police to reach a population that that they have greater exposure to than other first-responders. Due to their presence in areas marked by disadvantage, crime, and overdose mortality, it is likely that they have a greater likelihood of interacting with PWUOs compared to TFMR. 

Second, I will add to a body of literature that looks at officers’ attitudes towards carrying naloxone, administering naloxone, and people who use drugs (PWUDs) and people who use opioids (PWUOs) by examining how officers’ attitudes have changed from the inception of the ORP to the fourth and final year of the project. This is particularly informative for assessing long-term buy-in among patrol officers, specifically with regard to the compassion fatigue hypothesis, which argues that as officers respond to overdose calls over time they develop negative attitudes towards PWUDs and naloxone \parencite{carroll_knowledge_2020, murphy_police_2020, murphy_police_2021}.If police officers are to be a primary mechanism in administering naloxone and connecting overdose survivors with services, their perceptions of PWUDs, naloxone, and their role in responding to overdose incidents are critical for program effectiveness and sustainability. 

Lastly, using a quasi-experimental approach that compares Tempe’s opioid overdose fatality rate to another jurisdiction that did not receive the ORP intervention, I investigate the impact of the project on fatal opioid overdoses and total opioid overdose call volume. As many jurisdictions seek to develop a strategy to effectively reduce overdose mortality and opioid use disorder (OUD), Tempe provides a promising framework \parencite{white_moving_2021}. With opioid overdose mortality rates reaching all-time highs, the need for an effective strategy is quite salient. Up until this point, addressing the opioid overdose crisis has proven to be difficult given that demand-side, supply-side, and structural factors play a role \parencite{feldmeyer_community_2022}. This study will be one of only a handful to examine the impact of a comprehensive public safety-public health partnership to reduce opioid overdose fatalities.

\section{Dissertation's Research Context}

This dissertation is possible because of the Tempe First-Responder Opioid Recovery Project (ORP). The ORP is a police-led collaborative post-overdose outreach program between the Tempe Police Department (TPD) and La Frontera EMPACT --a behavioral/social services organization-- that connects opioid overdose survivors and their family members with behavioral and social services. This program, which is funded by the Substance Abuse and Mental Health Administration (SAMHSA), began in January 2020 in Tempe, Arizona, when police officers within the TPD began to be trained and outfitted with naloxone. In addition to being outfitted with naloxone, officers on scene at a suspected opioid overdose would initiate a clinician based post-overdose outreach response by calling a 24/7 hot-line with La Frontera EMPACT. The officer provides details of the incident such as the individual's name and the hospital that they are being transported to. Then, within a few hours, a peer navigator from La Frontera EMPACT makes contact with the individual at the hospital. The first contact in some cases was a co-response between the peer navigator and a police officer from TPD. This was the primary method with COVID-19 restrictions in place within hospitals as the TPD officer would gain access to the individual for the peer navigator to make contact. Once COVID-19 restrictions were lifted the co-response was less common. Following the first contact, typically within 24 hours, the peer navigator makes contact with the individual again to build rapport, discuss the project, and the services that are available. If services are accepted, the services' costs are covered for up to 45 days. 

Since the start of the program, Tempe police officers have responded to more than 323 opioid overdose incidents.\footnote{This is the most recent estimate from November, 2023.} Of the 323 overdose incidents, 294 resulted in a positive response to the naloxone dose(s) and the individuals recovered (91\%). Additionally, of those who were able to be contacted, approximately 54\% accepted services through La Frontera EMPACT which exceeds other engagement rates in similar programs \parencite{dahlem_beyond_2017, wagner_training_2016}. Services that were most commonly referred include residential treatment, re-connection to severely mental illness clinics, inpatient hospitalization, counseling, and intensive outpatient programs. Other frequent referrals included medications for opioid use disorder (MOUD), behavioral therapy programs, support groups, family support, employment support, and housing assistance.

\section{Dissertation Organization}

This dissertation is organized as follows. The next chapter focuses on the spatial context surrounding police and other first-responders involvement in opioid overdoses. Then, Chapter 3 is an evaluation of police officer perceptions and attitudes towards carrying and administering naloxone across multiple waves of surveys. Chapter 4 involves a quasi-experimental approach to evaluating the Tempe First-Responder Opioid Recovery Project. Lastly, Chapter 5 will entail a general discussion and conclusion of the three studies and how they inform the debate over police involvement in opioid related issues.
