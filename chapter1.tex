\chapter{The Opioid Crisis and Role of the Police}

% outline:
% discuss the opioid crisis - trends, drivers (theory, geography), costs, and responses (state, and local policy) lead to police being situated to respond
% BUT, there are debates about police involvement - criminalization (warrant checking, arrests), iatrogenic (drug seizures, naloxone stacking), fear of calling police among PWUOs, and broadly "police are law enforcers" (need cites)
% counter narrative re police - have been involved in public health issues since their inception (not just law enforcers), akin to a trained social worker, interact with vulnerable pops/PWUDs frequently, ....
% Lead into how this diss informs the debate surrounding police involvement in ODs
%
\section{Introduction}

The last two decades have been marked by a growing concern regarding opioid overdose trends, their impact on society, and the dearth of effective responses to combat this public health crisis. Since the 1990s, opioid overdose fatalities in the U.S. have increased exponentially. Opioid overdose fatalities reached an all-time high in 2021. Provisional estimates suggest a slight increase from 2021 to 2022 as well (82,998). According to the CDC, there were 80,411 fatal opioid overdoses in 2021 which eclipsed the previous high of 68,630 in 2020 \parencite{national_institute_on_drug_abuse_drug_2023}. These trends have been costly both socially and economically. In 2017, the overall monetary cost of the opioid overdose crisis was estimated to be \$1 trillion \parencite{luo_state-level_2021}. Given that opioid overdoses have reached unprecedented highs in recent years, this figure is likely to have increased. This public health crisis has led the Biden administration to award over \$1.6 billion in grants for investments into programmatic responses to address opioid overdoses in the U.S \parencite{us_department_of_health_and_human_services_biden-harris_2022}.

Overdose trends nationally, at the state-level, and locally have forced agencies and policymakers to consider approaches to effectively combat the ongoing crisis. Many of the approaches taken include various state laws (e.g., Good Samaritan Laws), prescription drug monitoring programs (PDMPs), naloxone access laws (NALs), insurance strategies, and others \parencite{haegerich_evidence_2019}. One such approach is collaboration between public safety and public health entities. Given that the police mission entails a wide variety of tasks, they frequently interact with people who use drugs (PWUDs) and respond to overdoses. Police are well-positioned to be a part of a response to opioid overdoses, including the provision of life-saving care such as the administration of naloxone. The police can then act as a conduit to services through referral or diversion mechanisms allowing public health agencies to provide information and help following contact with police.  Police may also facilitate a role within the context of post-overdose outreach with social worker or harm reduction specialists \parencite{bagley_scoping_2019}.

An important empirical finding that must be acknowledged in developing responses to the opioid overdose crisis is the spatial distribution of these events and the structural factors that are associated with overdoses, particularly overdose mortality Neighborhood-level structural characteristics such as concentrated disadvantage, income inequality, job loss, and population decline have been found to be associated with overdose mortality \parencite{carter_spatial_2019, feldmeyer_community_2022, ford_neighborhood_2017, piza_drug_2023}. Likewise, these neighborhood-level characteristics, particularly concentrated disadvantage is also associated with violent crime rates \parencite{peterson_divergent_2010}. The police are often operating within these areas of structural disadvantage due to the disproportionate spatial distribution of crime. A byproduct of this spatial concentration makes police work ecologically different from other first-responders. Not only do they outnumber other first-responders \parencite{lurigio_opioid_2018} but spatially, they are patrolling and often in areas where opioid overdoses are disproportionately concentrated. This aspect of police work may provide a practical foundation for officers to frequently interact with PWUDs, administer naloxone to those overdosing, and connect individuals with services in a way that other first-responders are unable to.

However, outfitting officers and other first-responders is not enough to combat the opioid overdose crisis as this approach is a short-term solution to reduce harm \parencite{goodison_law_2019}. Some jurisdictions have taken it a step further to incorporate social service organizations to become involved following police contact. Officers on scene at an overdose may act as a bridge to services either through referral or diversion in lieu of arrest \parencite{collins_seattles_2017, paul_meeting_2018}. Other programs may involve a post-overdose outreach team that combines both public safety and public health agencies to facilitate a response to those who previously overdosed. The social service organizations are then able to discuss with the individual who overdosed the services that are available and potentially helpful. In 2020, Tempe, Arizona began such a program, called the Tempe First-Responder Opioid Recovery Project (ORP). The ORP centers on a collaborative and multifaceted partnership between the Tempe Police Department and a social service provider, EMPACT, to address the opioid overdose crisis. 

The research on these collaborative approaches in addressing the opioid overdose crisis is lacking given it is a relatively new approach to addressing opioid overdoses. However, these approaches have been described as a promising path towards addressing the crisis \parencite{haegerich_evidence_2019, yatsco_alternatives_2020}. Post overdose outreach programs have been found to reduce fatal overdoses \parencite{xuan_association_2023}. Likewise, collaborative post-overdose outreach between police and public health entities suggest that the police can play an important role in reducing opioid overdoses \parencite{donnelly_law_2022}. The police mission, their proximity to the issue, and authority positions them to respond to opioid overdoses. Their ability to partner with public health organizations to reduce opioid overdose fatalities may provide a promising a path. Given the opioid overdose crisis has reached historic highs in recent years, it’s a pressing issue to develop effective responses. 

My dissertation will address the lack of empirical research on the impact of collaborative programmatic responses between public safety and public health entities in responding to the opioid overdose crisis. The first two studies of this proposed dissertation investigate important antecedents that may contribute to an effective collaborative response. First, I will provide descriptive statistics and visualizations of the spatial distribution of crime and opioid overdoses as well as TPD and TFMR naloxone administrations in Tempe, Arizona. Then, I will employ negative binomial regression models to assess the neighborhood-level predictors of TPD and TFMR naloxone administrations across block-groups. Examining opioid overdoses and naloxone administrations spatially provides insight into key aspects of the police being involved in opioid overdose incidents. Namely, the overlap of disadvantage, crime, and opioid overdoses provide a practical justification for the police to be involved in responding to overdoses, providing life-saving care, and administering naloxone given they are frequently in areas marked by high crime. Prior work has shown that EMS and police call for service data can differ spatially (Hibdon et al., 2017; Hibdon et al., 2021). This chapter will examine the block-group level socio-demographic factors that are associated with spatial variation in TPD and TFMR naloxone administrations. Outfitting police officers with naloxone and having them be a primary mechanism for the connection to social services (e.g., through referral, diversion, or co-response models) may allow the police to reach a population that that they have greater exposure to than other first-responders. Due to their presence in areas marked by disadvantage, crime, and overdose mortality, it is likely that they have a greater likelihood of interacting with PWUOs compared to TFMR. 

Second, I will add to a body of literature that looks at officers’ attitudes towards carrying naloxone, administering naloxone, and people who use drugs (PWUDs) and people who use opioids (PWUOs) by examining how officers’ attitudes have changed from the inception of the ORP to the fourth and final year of the project. This is particularly informative for assessing long-term buy-in among patrol officers, specifically with regard to the compassion fatigue hypothesis, which argues that as officers respond to overdose calls over time they develop negative attitudes towards PWUDs and naloxone (Carroll et al., 2020). If police officers are to be a primary mechanism in administering naloxone and connecting overdose survivors with services, their perceptions of PWUDs, naloxone, and their role in responding to overdose incidents are critical for program effectiveness and sustainability. 

Lastly, using a quasi-experimental approach that compares Tempe’s opioid overdose fatality rate to another jurisdiction that did not receive the ORP intervention, I investigate the impact of the project on fatal opioid overdoses and total opioid overdose call volume. As many jurisdictions seek to develop a strategy to effectively reduce overdose mortality and opioid use disorder (OUD), Tempe provides a promising framework \parencite{white_moving_2021}. With opioid overdose mortality rates reaching all-time highs, the need for an effective strategy is quite salient. Up until this point, addressing the opioid overdose crisis has proven to be difficult given that demand-side, supply-side, and structural factors play a role \parencite{feldmeyer_community_2022}. This study will be one of only a handful to examine the impact of a comprehensive public safety-public health partnership to reduce opioid overdose fatalities.

This dissertation is organized as follows. The next section will review the relevant literature on the opioid overdose crisis, trends in overdoses as well as theoretical explanations for these trends, and efforts to combat rising fatal opioid overdoses. Then, I will delve into the debate surrounding the proper role of the police in handling public health issues and more specifically, opioid related issues. This leads into the three studies of this dissertation that will inform this debate by examining a few aspects of the police role in opioid overdoses. %describe each chapter briefly here?


\section{Background}
\subsection{Overview of the Opioid Overdose Crisis}

%In 2021, there were 106,699 drug overdose fatalities which accounted for more deaths than motor vehicle fatalities (45,404) and all firearm deaths combined (48,830)(Center for Disease Control and Prevention [CDC], 2023b). Drug overdose is one of the leading causes of accidental death in the U.S., and for those aged 1 to 44, it is the leading cause of accidental death. Dying from a drug overdose used to be relatively rare (CDC, 2023a). The drug overdose death rate hovered around 2 per 100,000 in the 1980s and early 1990s (M. Warner et al., 2011). By 2000, it was over 5 per 100,000 and a decade later it was over 10 per 100,000 (Rudd et al., 2016). In 2020, it roughly tripled to just over 28 per 100,000 (Hedegaard et al., 2021). This increase in drug overdoses has been driven by the use of opioids, and as of late, synthetic opioids (e.g., fentanyl and its analogs). The exponential growth of fatal opioid overdoses has been descriptively categorized into three notable waves. The first wave began in the 1990s, driven primarily by supply side mechanisms related to the over prescription of painkillers (Kolodny et al., 2015). Other demand-side factors contributed to the uptick in opioid usage as well such as economic despair, aging populations, etc. (Dasgupta et al., 2018). The second wave of opioid overdoses was a shift away from prescription medications to heroin as the former became increasingly regulated. Younger individuals’ use of opioids declined while heroin use increased, suggesting a form of dependency and a transition into a new market (Unick & Ciccarone, 2017). The third wave is marked by the introduction of synthetic opioids (i.e., fentanyl) into the drug markets around 2014. Fentanyl’s emergence into the market in 2014 is suggestive of supply-side innovation, namely because fentanyl is cheaper than heroin. There wasn't much demand for fentanyl early on, although the demand for it grew over time (Powell & Pacula, 2020). Scholars who have conducted ethnographic work have found that users vary in their desire to use fentanyl (Carroll et al., 2017; Mars et al., 2019). Additionally, because of its potency (Ciccarone, 2017) and its prevalence in other substances – such as methamphetamine, cocaine, and benzodiazepines – fentanyl has drastically impacted the mortality rates associated with overdoses for a variety of drug types, particularly cocaine (Humphreys et al., 2022). 

The cost of a fatal overdose is estimated to be \$11.5 million (e.g., loss of life, health care, loss of economic productivity), while the estimate for a non-fatal overdose is \$2,651 (e.g., naloxone and hospital costs) (see CDC, 2020). Using recent estimates from the CDC, there were 80,411 drug overdose fatalities that involved an opioid in 2021. Based on back-of-the-envelope calculations, the cost of 80,411 opioid overdose fatalities is roughly \$924.7 billion, a 68\% increase from 2017 estimates (\$549.69 billion). This increase in cost highlights the immediate need to evaluate programs and policies devised to combat OUD and reduce the mortality of opioid overdoses. 

\subsection{Efforts to Curb OUD and Fatal Overdoses}
\subsubsection{State and Local Policy}

Many changes geared towards combatting the opioid overdose crisis have come from state legislation. Some of the notable legislative changes at the state level include Good Samaritan Laws, naloxone access laws (NALs), naloxone education and distribution, state PDMPs (Haegerich et al., 2019), and fentanyl test strip distribution. Some of the earliest efforts to reduce overdose mortality at the state-level come from New Mexico when, in 2001, the state enacted the first naloxone access law which provided immunity to those who administered naloxone on someone who was experiencing an opioid overdose. The law also allowed for the distribution of naloxone without legal consequences. Just 6 years later, New Mexico legislation signed off on a Good Samaritan Law which provides immunity to individuals who call for medical assistance at the scene of an overdose as well as those who experience the overdose (Rees et al., 2019). Since New Mexico’s early adoption of these laws, other states have followed suit. However, the empirical base for NALs reducing fatal overdoses is generally mixed (Smart et al., 2021). But the mixed evidence is likely to be attributable to the varying components of NALs. For instance, some states that implement a NAL have a standing order, remove criminal liability for naloxone possession, remove prescriber immunity, and/or allow third-party prescriptions. Often, a NAL that is passed stipulates more than one of the described components above. Empirically speaking, McClellan et al. (2018) and Reese et al. (2019) found statistically significant decreases in overdose mortality associated with the passage of NALs. Reese et al. (2019) finds heterogenous effects between the components of the NALs. For instance, removal of criminal liability for possession of naloxone was associated with lower opioid and non-opioid overdose mortality, but not heroin overdose mortality. Prescriber immunity was associated with lower opioid and heroin overdose mortality, but not non-opioid overdose mortality. However, other studies suggest a null effect of NALs (Abouk et al., 2019; Atkins et al., 2019; Doleac & Mukherjee, 2018), although one study reported an increase in overdose mortality (Erfanian et al., 2019). 

Prescription drug monitoring programs consist of a state-level database that tracks the number of prescriptions being written by doctors. This policy has been shown to reduce opioid prescribing (Bao et al., 2016; Haegerich et al., 2014), however, with synthetic opioids driving overdose mortality this policy is likely to have a smaller impact now than in the early years of the opioid overdose crisis. To combat this issue, fentanyl test strips have been dispensed to PWUOs to ensure the safety of their drug supply. Evidence suggests that individuals use the test strips and alter their use practices to either use less or utilize further safety precautions (Park et al., 2021; Peiper et al., 2019).
PAARI Programs 

Collaboration between public safety and public health agencies has become a more common approach to address the opioid overdose crisis, because of the multifaceted nature of the problem. Partnerships between public safety and public health have been a foundation for programmatic responses to address issues such as interpersonal violence, juvenile delinquency, mental illness, alcohol and substance use, and domestic violence (Patterson & Swan, 2019). More recently, collaboration has been highlighted in Law Enforcement Assisted Diversion programs (LEAD) programs which have been implemented in various jurisdictions across the United States in an attempt to divert low level offenders away from the criminal justice system and provide social services that are able to help the individual (see Collins et al., 2017; Perrone et al., 2022). Moreover, this collaborative response to low-level offenses has been deemed a promising approach (National Institute of Justice, 2016).

The Police Assisted Addiction and Recovery Initiative (PAARI) was devised as a response to divert PWUDs away from the criminal justice system in hopes to engage individuals with a range of social services, from treatment options to other services such as obtaining a driver’s license (Goodison et al., 2019). This approach is similar to LEAD yet more narrow with respect to the targeted population. PAARI programs are also not solely diversion programs. Diversion can be a mechanism in some cases; however, the program may utilize outreach workers in an attempt to offer information on services that may be of use to those who overdosed. Programs where officers or outreach workers engage PWUDs and refer them to services are often coined as “referral” programs. This is often the case with post-overdose outreach programs that seek to contact the individual who overdoses in the coming hours or even days after the non-fatal overdose (Bagley et a., 2019). However, the diversion/referral mechanisms are not mutually exclusive and can both be a part of the intervention (see Donnelly et al., 2022).

The evidence base surrounding these collaborative programs is promising, although limited. Many of the studies that evaluate collaborative approaches to the opioid overdose crisis have focused on outputs such as the number of naloxone kits distributed, naloxone administrations, survey data, and the number of overdose survivors engaged in services (see Bailey et al., 2023). These outputs are critically important milestones for any program, but they do not measure the central outcome of interest: overdoses. To my knowledge, two studies have evaluated the change in fatal overdoses, non-fatal overdoses, or opioid overdose calls for service following a collaborative intervention between public safety and public health entities. Xuan et al. (2023) use an interrupted time series model comparing municipalities in Massachusetts that adopted post-overdose outreach program, compared to those that did not. They find that post-overdose outreach was associated with a yearly decrease in the opioid overdose fatality rate of -0.43 per 100k. EMS response rate also decrease yearly at a rate of -5.14 per 100k. Donnelly et al. (2022) use a dynamic forecasting model to evaluate a law enforcement-based outreach/referral program’s impact on non-fatal and fatal overdoses. They find that monthly observed fatal and non-fatal overdoses were less than the predicted values. Specifically, they found that in the pooled post-intervention period there were 1.85 fewer fatal overdoses and 7.25 fewer non-fatal overdoses per month which equated to a total of 85 fewer fatal and 333 fewer non-fatal overdoses in the post-intervention period. Both articles suggest the potential effectiveness of collaborative post-overdose outreach.

\section{The Role of the Police}

The police are constantly called upon to handle a host of societal problems. Quite frequently, the problems that police deal with are not crime related (Mastrofski, 1983; Wilson, 1968). Indeed, the institution of policing can be thought of as a last resort entity that handles societal issues that seem to permeate through other institutional failures (Bittner, 1967). It follows that the role of police in society is not simply to control crime or apply the law in a strict manner (Goldstein, 1990). Order maintenance, for instance, is a role of the police that does not strictly rely on arrests as a means to fulfill duties. 

The early police forces in the northern cities of America embodied this order maintenance role as 'watchmen' who did not necessarily rely on the law or arrest to keep peace (CITES). This held true through the early 20th century (Monnekon?). It wasn't until the professionalization movement gained traction that the order maintenance role subsided in popularity. Despite the decline in popularity, the concept of order maintenance policing experienced a revitalization with Wilson and Kelling's (YEAR) discussion of broken windows policing. The authors argue that the police act a as a control mechanism for the community to prevent further disorder from accumulating and to limit the amount of crime that stems from disorder. Wilson and Kelling's depiction of the police suggest that they are not solely law enforcers and can create order and peace through non-enforcement methods. Although, the theoretical framework of broken window's policing was used as the foundation of an enforcement heavy strategy (CITES; SQF). Nonetheless, the order maintenance role of the police has historically been associated with a non-law enforcement based approach that emphasizes the role of informal control. Within this framework, police officers’ day-to-day activities are more reminiscent of a social worker or a street corner politician (Muir, 1977). 

However, there are arguments that the police mandate must be constrained. This sentiment is echoed by both police critics and the police themselves in some cases (see Hodges, 2017). Some argue that the police should be abolished (Vitale, 2017), others argue that their function must be restricted to more specific domains. Nonetheless, when the police are responding to and attempting to resolve issues in the community that range from mundane disputes between neighbors to murder, it is evident that there is a wide range of societal problems that fall into the police domain. Egon Bittner famously stated that "no human problem exists, or is imaginable, about which it could be said with finality that this certainly could not become the proper business of the police” (1970, p. 244). This quote not only highlights the broad complexity of the police mission, which has been described as “the impossible mandate” (Manning, 1978), but it captures the fact that the police respond to and attempt to resolve a host of community problems which includes public health issues.

Frequently, the police are called upon by citizens to resolve whatever issue may be at hand. This could involve a violation of the law – which is precisely the job of the police. However, in some instances, these issues do not necessitate the enforcement of the law by an officer and may indeed entail more of a public health response, or at a minimum, a non-criminal justice response. A practical reason the police respond to and handle public health issues is that their role in society is much broader than the enforcement of the law and their day-to-day activities are often intertwined with public health problems (Wood, 2020). Additionally, calls for service to crime-related events may turn out to be health related (Ratcliffe, 2021). A more philosophical reason the police handle public health issues is that if the police were to be described as having a single mission, it would be to protect the citizenry as they are called upon to resolve human problems that may require the use of force (Bittner, 1970; Skolnick & Fyfe, 1993). Responding to public health issues falls into the purview of the police based on Bittner’s prominent depiction of the role of the police in society.

Additionally, the police provide time sensitive life-saving measures to those in need. Examples include someone who is in a burning car, or house; someone who is drowning; animals that need rescuing; debris that has someone trapped. It's fair to say that these examples are not a part of the typical schema that one has when they envision the police. But it is also fair to say most individuals do not have a problem when a police officer provides life-saving measures in any of the instances described above. Why is an opioid overdose any different? Below I discuss the issues surrounding police involvement in substance use and opioid use specifically. I then return to this question discussing some of the reasons police ought to respond to opioid overdoses. 
 
\subsection{Police Involvement in Substance Use}

Police involvement in drug-related issues has been contentious as U.S. law enforcement has historically relied on arrests and incarceration to reduce harm and create safe communities. The focus on arrests and incarceration embodies supply-side strategies which include drug busts, undercover operations, and an emphasis on drug task forces. This approach was the product of political maneuvering to create a War on Drugs which began in 1973 under President Nixon and was reified under President Reagan in 1982 (Cooper, 2015; Lynch, 2012).

To further emphasize the police role with respect to drug use, both federal- and state-level legislation reified the notion that drug use was a public safety issue with harsh penalties for those in possession of or selling narcotics. Examples include the New York Rockefeller Drug Laws (Drucker, 2002) and the disparate sentencing for crack cocaine versus cocaine (Shein, 1993). Policies were implemented that incentivized police to arrest those suspected of drug possession or selling with the passage of the 1984 Omnibus Crime Bill. This allowed police to confiscate any assets believed to have been purchased with money obtained via the drug market (Beckett & Sasson, 2003). Drug arrests increased through the 1980s and 1990s with a disparate impact on minority communities (Western, 2006). The enforcement focused approach that the police have occupied when addressing drug-related issues has been more costly than beneficial (Baum, 1996). 

However, the opioid overdose crisis has a different climate surrounding it than prior drug crises. Fedders (2019) describes the shift in police work surrounding the opioid overdose crisis as opioid policing while others have referred to the new role as harm-reduction policing (see Beckett, 2016). The emphasis on treatment and harm-reduction has been the focus of policymakers for the last two decades during the growing opioid overdose crisis. The purported reasons for the shift from enforcement to public health focused vary. Some argue that this is a result of the stereotypical opioid user being white and from suburban/rural areas as opposed to Black, urban dwellers (Hart & Hart, 2019). Although, the demographics of those fatally overdosing from opioids has shifted in recent years (see Humphreys et al., 2022). Further explaining the shift in approaches, others point to the fact that the opioid overdose crisis is a product of greedy pharmaceutical companies who aggressively marketed Oxycontin to physicians and provided illegal payments to physicians in exchange for writing Oxycontin prescriptions (Hoffman & Benner, 2020). Moreover, it was the prescription of painkillers that played a large role early in the crisis (Dasgupta et al., 2018). The latter two explanations arguably remove some culpability away from the user, which may be associated with a move towards treatment. And more broadly, the War on Drugs of the 1980s and 1990s is seen not only as costly, but a failure. Thus, the enforcement heavy approach to drug related issues was altered to include a public health model. 

\subsubsection{Public Health Experts’ Concerns} %Expand on the debate over the role of the police

Returning to the question of, 'Why is an opioid overdose different?', here I will discuss some of the reasons why an opioid overdose is different than the other emergency situations described above. Broadly, the issue of police involvement in opioid overdoses is the concept of iatrogenesis. Police officers who respond to opioid overdoses and intend on helping, end up causing harm. For instance, when officers arrive on scene at an opioid overdose and engage in warrant checking and arrest individuals on scene is - in the eyes of the police - good police work. It's not problematic in and of itself that officers check for warrants and make arrests. After all, this is a part of their role in society. It's problematic because it is based on a flawed idea that solving issues related to drug possession, and more specifically the opioid crisis, is contingent upon police arresting overdose survivors and/or people on scene at the overdose . Plus, this approach has cascading effects that actually cause harm such as an increased likelihood of fatal overdose.

When police engage in warrant checking and arrest the overdose survivor, 

Reduced tolerance after incarceration, increase likelihood of fatal od
Reduced likelihood of citizens calling 911, increase likelihood of fatal od


The cultural, political, and programmatic shift to focus on treatment and harm-reduction has influenced policy at both the federal- and state-level. The shift towards a public health model for police has primarily entailed diversion and referral programs to get PWUDs in contact with social services. However, these changes can be hampered when police are involved. Officer discretion can prevent policy change from being enacted at the street-level (Friedman et al., 2021). Police officers are a mediator between the law on the books and the law in practice – they are the brokers of the rules of social cooperation (del Pozo, 2022). As an example, Good Samaritan Laws provide immunity for the individual who overdosed and others on scene who call for help and administer aid. However, officers have the discretion to arrest for outstanding warrants, and other offenses at the scene of the overdose. To address this issue, some have argued for a shift in the goal of the police, particularly as it applies to the policing of drugs. del Pozo et al. (2021) argue for public health ethics within police departments which would translate to a holistic approach focused on the health and general well-being of citizens. Survey research has indicated that some officers see the value in adopting harm reduction methods that focus on the health of citizens, particularly as it pertains to the carrying of naloxone and collaboration with outreach organizations (Banta-Green et al., 2013; Lloyd et al., 2023; White, et al., 2021). And body-worn camera footage has shown that officers infrequently arrest overdose survivors and can act as a conduit to service providers (White et al., 2022). However, there are concerns that both police officers and public health experts have regarding the police being involved in opioid-related issues.



Public health experts have expressed concerns over police involvement in substance use issues, particularly opioid use, primarily due to the criminalization of PWUDs (Carroll et al., 2021; Van Der Meulen et al., 2021) but also due to the iatrogenic effects of police involvement, such as stacking naloxone doses which may produce withdrawal effects. Prior research has found that the likelihood of arrest given a police response is higher compared to receiving an emergency medical services (EMS) response (Lowder et al., 2020). This finding in addition to the history of police involvement in drug issues has resulted in those on scene at an overdose reporting that they are fearful of a police response resulting in an arrest if they call 911 (Van Der Meulen et al., 2021). Like the public health issues described above, PWUDs will likely not benefit from incarceration. In fact, PWUDs’ likelihood of fatal overdose is heightened following incarceration due to their tolerance levels lowering (Binswanger et al., 2016; Merrall et al., 2010). This concern has been replicated within the context of police tactics in a recent study by Ray et al. (2023) that highlights the spatio-temporal effects of drug busts on overdoses. They find that following opioid-related seizures, non-fatal overdoses, and fatal overdoses, naloxone administrations increased at 7, 14, and 21 days within a distance of 100, 250, and 500 meters, respectively. While this study does not purport a causal claim, the authors note the theoretical underpinnings are akin to the logic behind the post-release incarceration and increased fatal overdose likelihood (Seaman et al., 1998). Mainly, following a drug bust, users may seek out new dealers which comes with risks, particularly the uncertainty regarding the safety of the dose from a new supplier and unknown tolerance levels (Ray et al., 2023). This pioneering work as well as the other research on the iatrogenic effects of police at the scene of overdoses highlights the debate over their role in substance use.
Police Concerns

The police have expressed their own concerns about responding to drug overdoses, particularly regarding carrying and administering naloxone. These concerns include the cost of naloxone (Jamison, 2019), potential liability (Banta-Green et al., 2013), and potential exposure to opioids (del Pozo et al., 2021). Some studies find that officers have generally negative perceptions of PWUDs (Smyser & Lubin, 2018; Winograd et al., 2019). Officers have also indicated feeling conflicted between public safety and public health orientations when dealing with opioid use (Banta-Green et al., 2013). Support from officers has also been noted. Officers have reported that carrying naloxone improves their ability to handle overdose situations and they are generally glad to be carrying it (White et al., 2021) and that the traditional approach to substance use is likely ineffective and are more apt to support collaborative approaches (Lloyd et al., 2023). This highlights officers’ support for the role that outreach workers and other public health entities can have following an overdose incident. The perspectives of officers are important to analyze given their frequent contact with PWUDs and potential role in responding to opioid-related issues. 

Despite the concerns that have been raised by both the police and public health experts, in recent years some jurisdictions have tasked their police officers with approaching overdose incidents with a public health orientation through co-response models, diversion, or post-overdose outreach programs. Given this shift from supply-side to a more demand-side focused approach, officers’ attitudes, and perceptions about their role in this shift are important to investigate.

\section{Remaining questions}
	
Despite the abundance of research on the opioid overdose crisis and programmatic responses, police officer attitudes towards PWUDs and naloxone, and the spatial distribution of overdoses, there are still important questions that remain unanswered. First, why should the police be involved in responding to opioid overdoses? Due to police outnumbering other first-responders, their availability, and their patrolling patterns, the police are positioned to effectively respond to opioid overdoses. Because of these characteristics of police work, it may be the case that police officers respond to areas more frequently than emergency medical services (EMS). If officers are outfitted with naloxone, the geographic distribution of naloxone administrations may differ compared to EMS. Both Hibdon et al. (2017) and Hibdon et al. (2021) find that the geographical distribution of calls for service between EMS and the police department overlap some but also vary. This finding is true for drug activity and violence, suggesting that the populations that the police department and EMS interact with are likely different. I will explore the spatial distribution between Tempe PD and TFMR’s naloxone administrations to see if police are responding to and administering naloxone in certain areas of the city more frequently than TFMR. Then, I will analyze the neighborhood-level characteristics that are associated with the spatial distribution of TPD and TFRM naloxone administrations. For the purposes of collaborative approaches to the opioid overdose crisis, officers administering naloxone in certain areas more frequently than EMS suggests the ability to provide life-saving medication in areas that EMS may not be responding to as quickly as a police department. Additionally, it may provide an important bridge to services in these areas. However, this has yet to be explored.

Second, despite the police having a unique role in society that positions them to respond to opioid overdoses, do police officers want to be involved in responding to opioid overdoses? More specifically, do officers develop negative attitudes towards PWUDs, naloxone, and their role over time? These attitudes are crucial to the potential for sustainable partnerships. Most studies that look at officer’s perceptions are captured at one point in time. And previous studies have found an association between increased overdose response and negative attitudes, which is suggestive of a compassion fatigue effect (Carrol et al., 2020). However, officer perceptions have not been assessed over time through multiple waves of surveys from the beginning to the end of a police-social work collaboration to address opioid overdoses. The compassion fatigue hypothesis has yet to be investigated in more detail. For instance, aggregate trends in officer attitudes across multiple waves of surveys has yet to be explored. I will explore Tempe police officer attitudes towards PWUDs, naloxone, and their role in the opioid overdose crisis with data from multiple waves of surveys during a four-year collaborative approach with a social services agency to reduce opioid overdose fatalities. 

Lastly, does a collaborative response between public safety and public health entities work to reduce opioid overdose fatalities? Because of the intersection between police and public health, particularly as it relates to opioid use, programs that involve the police should be evaluated given that police agencies are carrying naloxone and increasingly developing responses that are focused on non-arrest alternative methods. However, there is very little research evaluating the effectiveness of collaborative approaches between public safety and public health organizations in combatting opioid overdose fatalities (Yatsco et al., 2020). Two studies provide a rigorous (e.g., quasi-experimental) evaluation of the impact of post-overdose outreach programs in reducing opioid overdose fatalities (see Donnelly et al., 2022; Xuan et al., 2023). I will add to this limited body of research by using a controlled interrupted time series analysis to investigate the relationship between the intervention and opioid overdose trends.
