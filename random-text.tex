\subsection{Efforts to Curb OUD and Fatal Overdoses}
\subsubsection{State and Local Policy}

Many changes geared towards combating the opioid overdose crisis have come from state legislation. Some of the notable legislative changes at the state level include Good Samaritan Laws, naloxone access laws (NALs), naloxone education and distribution, state PDMPs (Haegerich et al., 2019), and fentanyl test strip distribution. Some of the earliest efforts to reduce overdose mortality at the state-level come from New Mexico when, in 2001, the state enacted the first naloxone access law which provided immunity to those who administered naloxone on someone who was experiencing an opioid overdose. The law also allowed for the distribution of naloxone without legal consequences. Just 6 years later, New Mexico legislation signed off on a Good Samaritan Law which provides immunity to individuals who call for medical assistance at the scene of an overdose as well as those who experience the overdose (Rees et al., 2019). Since New Mexico’s early adoption of these laws, other states have followed suit. However, the empirical base for NALs reducing fatal overdoses is generally mixed (Smart et al., 2021). But the mixed evidence is likely to be attributable to the varying components of NALs. For instance, some states that implement a NAL have a standing order, remove criminal liability for naloxone possession, remove prescriber immunity, and/or allow third-party prescriptions. Often, a NAL that is passed stipulates more than one of the described components above. Empirically speaking, McClellan et al. (2018) and Reese et al. (2019) found statistically significant decreases in overdose mortality associated with the passage of NALs. Reese et al. (2019) finds heterogenous effects between the components of the NALs. For instance, removal of criminal liability for possession of naloxone was associated with lower opioid and non-opioid overdose mortality, but not heroin overdose mortality. Prescriber immunity was associated with lower opioid and heroin overdose mortality, but not non-opioid overdose mortality. However, other studies suggest a null effect of NALs (Abouk et al., 2019; Atkins et al., 2019; Doleac \& Mukherjee, 2018), although one study reported an increase in overdose mortality (Erfanian et al., 2019). 

Prescription drug monitoring programs consist of a state-level database that tracks the number of prescriptions being written by doctors. This policy has been shown to reduce opioid prescribing (Bao et al., 2016; Haegerich et al., 2014), however, with synthetic opioids driving overdose mortality this policy is likely to have a smaller impact now than in the early years of the opioid overdose crisis. To combat this issue, fentanyl test strips have been dispensed to PWUOs to ensure the safety of their drug supply. Evidence suggests that individuals use the test strips and alter their use practices to either use less or utilize further safety precautions (Park et al., 2021; Peiper et al., 2019).
PAARI Programs 

Collaboration between public safety and public health agencies has become a more common approach to address the opioid overdose crisis, because of the multifaceted nature of the problem. Partnerships between public safety and public health have been a foundation for programmatic responses to address issues such as interpersonal violence, juvenile delinquency, mental illness, alcohol and substance use, and domestic violence (Patterson \& Swan, 2019). More recently, collaboration has been highlighted in Law Enforcement Assisted Diversion programs (LEAD) programs which have been implemented in various jurisdictions across the United States in an attempt to divert low level offenders away from the criminal justice system and provide social services that are able to help the individual (see Collins et al., 2017; Perrone et al., 2022). Moreover, this collaborative response to low-level offenses has been deemed a promising approach (National Institute of Justice, 2016).

The Police Assisted Addiction and Recovery Initiative (PAARI) was devised as a response to divert PWUDs away from the criminal justice system in hopes to engage individuals with a range of social services, from treatment options to other services such as obtaining a driver’s license (Goodison et al., 2019). This approach is similar to LEAD yet more narrow with respect to the targeted population. PAARI programs are also not solely diversion programs. Diversion can be a mechanism in some cases; however, the program may utilize outreach workers in an attempt to offer information on services that may be of use to those who overdosed. Programs where officers or outreach workers engage PWUDs and refer them to services are often coined as “referral” programs. This is often the case with post-overdose outreach programs that seek to contact the individual who overdoses in the coming hours or even days after the non-fatal overdose (Bagley et a., 2019). However, the diversion/referral mechanisms are not mutually exclusive and can both be a part of the intervention (see Donnelly et al., 2022).

The evidence base surrounding these collaborative programs is promising, although limited. Many of the studies that evaluate collaborative approaches to the opioid overdose crisis have focused on outputs such as the number of naloxone kits distributed, naloxone administrations, survey data, and the number of overdose survivors engaged in services (see Bailey et al., 2023). These outputs are critically important milestones for any program, but they do not measure the central outcome of interest: overdoses. To my knowledge, two studies have evaluated the change in fatal overdoses, non-fatal overdoses, or opioid overdose calls for service following a collaborative intervention between public safety and public health entities. Xuan et al. (2023) use an interrupted time series model comparing municipalities in Massachusetts that adopted post-overdose outreach program, compared to those that did not. They find that post-overdose outreach was associated with a yearly decrease in the opioid overdose fatality rate of -0.43 per 100k. EMS response rate also decrease yearly at a rate of -5.14 per 100k. Donnelly et al. (2022) use a dynamic forecasting model to evaluate a law enforcement-based outreach/referral program’s impact on non-fatal and fatal overdoses. They find that monthly observed fatal and non-fatal overdoses were less than the predicted values. Specifically, they found that in the pooled post-intervention period there were 1.85 fewer fatal overdoses and 7.25 fewer non-fatal overdoses per month which equated to a total of 85 fewer fatal and 333 fewer non-fatal overdoses in the post-intervention period. Both articles suggest the potential effectiveness of collaborative post-overdose outreach.

\section{Remaining questions}
	
Despite the abundance of research on the opioid overdose crisis and programmatic responses, police officer attitudes towards PWUDs and naloxone, and the spatial distribution of overdoses, there are still important questions that remain unanswered. First, why should the police be involved in responding to opioid overdoses? Due to police outnumbering other first-responders, their availability, and their patrolling patterns, the police are positioned to effectively respond to opioid overdoses. Because of these characteristics of police work, it may be the case that police officers respond to areas more frequently than emergency medical services (EMS). If officers are outfitted with naloxone, the geographic distribution of naloxone administrations may differ compared to EMS. Both Hibdon et al. (2017) and Hibdon et al. (2021) find that the geographical distribution of calls for service between EMS and the police department overlap some but also vary. This finding is true for drug activity and violence, suggesting that the populations that the police department and EMS interact with are likely different. I will explore the spatial distribution between Tempe PD and TFMR’s naloxone administrations to see if police are responding to and administering naloxone in certain areas of the city more frequently than TFMR. Then, I will analyze the neighborhood-level characteristics that are associated with the spatial distribution of TPD and TFRM naloxone administrations. For the purposes of collaborative approaches to the opioid overdose crisis, officers administering naloxone in certain areas more frequently than EMS suggests the ability to provide life-saving medication in areas that EMS may not be responding to as quickly as a police department. Additionally, it may provide an important bridge to services in these areas. However, this has yet to be explored.

Second, despite the police having a unique role in society that positions them to respond to opioid overdoses, do police officers want to be involved in responding to opioid overdoses? More specifically, do officers develop negative attitudes towards PWUDs, naloxone, and their role over time? These attitudes are crucial to the potential for sustainable partnerships. Most studies that look at officer’s perceptions are captured at one point in time. And previous studies have found an association between increased overdose response and negative attitudes, which is suggestive of a compassion fatigue effect (Carrol et al., 2020). However, officer perceptions have not been assessed over time through multiple waves of surveys from the beginning to the end of a police-social work collaboration to address opioid overdoses. The compassion fatigue hypothesis has yet to be investigated in more detail. For instance, aggregate trends in officer attitudes across multiple waves of surveys has yet to be explored. I will explore Tempe police officer attitudes towards PWUDs, naloxone, and their role in the opioid overdose crisis with data from multiple waves of surveys during a four-year collaborative approach with a social services agency to reduce opioid overdose fatalities. 

Lastly, does a collaborative response between public safety and public health entities work to reduce opioid overdose fatalities? Because of the intersection between police and public health, particularly as it relates to opioid use, programs that involve the police should be evaluated given that police agencies are carrying naloxone and increasingly developing responses that are focused on non-arrest alternative methods. However, there is very little research evaluating the effectiveness of collaborative approaches between public safety and public health organizations in combatting opioid overdose fatalities (Yatsco et al., 2020). Two studies provide a rigorous (e.g., quasi-experimental) evaluation of the impact of post-overdose outreach programs in reducing opioid overdose fatalities (see Donnelly et al., 2022; Xuan et al., 2023). I will add to this limited body of research by using a controlled interrupted time series analysis to investigate the relationship between the intervention and opioid overdose trends.

The next section will review the relevant literature on the opioid overdose crisis and efforts to combat rising fatal opioid overdoses. Then, I will delve into the debate surrounding the proper role of the police in handling opioid related issues. The debate around the role of the police provides the foundation for this dissertation. I will round off the next section by providing an overview of this dissertation's research context which will involve a description of the Tempe First-Responder Opioid Recovery Project and Tempe, Arizona as a study site. I will not repeat this description in the three studies that follow to be concise but note that this is the study context that informs all three studies.