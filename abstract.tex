The police have a broad mission which entails responding to a variety of crime- and public health-related calls for service. Rising opioid overdose mortality rates have generated debates about how the police fit into this issue. Specifically, what are the police to do about rising opioid overdose mortality? This dissertation focuses on this broad question and presents three studies that inform the debates surrounding police officer involvement in the opioid overdose crisis. Study 1 uses a spatial point pattern test to examine spatial variation in where police and Tempe Fire and Medical Rescue (TFMR) are administering naloxone and mixed-effects regression models to assess the incident- and neighborhood-level characteristics that are associated with police officers being first on scene to administer naloxone at an opioid overdose incident. The results of study 1 suggest that police and TFMR are administering naloxone first at an opioid overdose in different neighborhoods. Moreover, both incident and neighborhood characteristics explained variation in when and where police and TFMR were administering naloxone first. Specifically, compared to TFMR, police officers were more likely to administer naloxone first in neighborhoods with higher proportion of White and Black residents, and higher drug offense rates. Police officers were less likely than TFMR to administer naloxone first in neighborhoods with higher violent crime rates, and when the incident was dispatched as mental health related, or other, and when the individual on scene was aged 40 to 59. Study 2 focuses on Tempe police officers' attitudes towards their role in the opioid overdose crisis, naloxone, and people who use opioids (PWUO) across 4 waves of surveys. Importantly, I focus on the compassion fatigue hypothesis to assess if officers attitudes change over time and if opioid overdose response frequency is associated with changes in attitudes. The results suggest that police officers in Tempe both support their role in the opioid overdose crisis but yet, negative perceptions towards PWUO increased over time. Though, opioid overdose response frequency is not associated with any of the outcomes. Study 3 is an impact evaluation of the Tempe First-Responder Opioid Recovery Project (ORP). I employ a comparative interrupted time series design to examine the impact of the project on the opioid overdose calls for service rate and fatal opioid overdose rate. The results suggest that the Tempe ORP did not have an impact on either outcome. The findings from this dissertation inform the debates around the role of the police and have implications for both theory and policy.