\chapter{Collaborative Response to the Opioid Overdose Crisis: Evidence from a Quasi-Experimental Analysis}

\section{\centering Introduction}
- opioid overdose crisis

- approaches to reduce OUD/overdoses

- police involvement in referrals

    - limited empirical work in this area (gap)

- brief: present study

\section{\centering Literature Review}

\subsection{Police Role in Reducing Overdose Fatalities}

The police have a broad mandate primarily encapsulated by the duty to protect lives \parencite{skolnick_above_1993}. This broad goal can appear in many different ways. As community problems shift over time, so do the methods that the police use to address them. One notable community problem that arose in the late 1990s and has caused exponential harm since, is the opioid overdose crisis. 

In an effort to address the harms created by the opioid overdose crisis, police began carrying naloxone, an opioid antagonist that binds to the opioid receptors in the brain and blocks the opioid from binding \parencite{lurigio_opioid_2018}. Given police frequently interact with people who use opioids and are on scene at overdoses \parencite{beletsky_police_2011, silverman_harmonizing_2012, white_leveraging_2022}, it is important to consider the utility of police carrying naloxone. Importantly, it fits within their mission to protect lives. Recent estimates suggest that approximately 82\% of police departments across the U.S. carry naloxone \parencite{ray_national_2023}. Moreover, naloxone is a short-term solution to a much more pervasive crisis. It is a harm reduction tool that can saves lives \parencite{rando_intranasal_2015, rees_little_2019}. However, more recently, in some jurisdictions police involvement in the opioid overdose crisis has become more involved through diversion or referral methods, or collaboration with public health agencies in an effort to provide treatment options to people who have overdosed. These approaches have been adopted because of the efforts of public health agencies to limit contact with the criminal justice system and police officers recognizing the importance of relying on alternative methods to arrest \parencite{e.g., Aitken et al., 2002; Bluthenthal et al., 1999; Flath et al., 2019; Friedman et al., 2006; Reuter, 2009; Small et al., 2006}

\subsubsection{Law Enforcement Assisted Diversion (LEAD)}

One such collaborative approach is Law Enforcement Assisted Diversion (LEAD), which involves a host of local agencies and criminal justice actors, and centers around law enforcement officers offering services to low level offenders (e.g., sex work, violation of drug laws) in lieu of arrest \parencite{national_institute_of_justice_program_2016}. Originally implemented in Seattle in 2011, this approach has expanded to other cities in an effort to limit contact with the criminal Justice system. With police officers having wider discretion to arrest for low-level offenses, the focus here is to restrict their discretion and promote a path towards behavioral or social services. Importantly, the evidence suggests LEAD is an effective program.

The evaluation of the Seattle program found that felony charges and arrests declined 57\% \parencite{collins_seattles_2017} and that employment and housing was more likely to be attained \parencite{clifasefi_seattles_2017}. A more recent evaluation of LEAD in San Francisco produced similar findings with a lower likelihood of felony and misdemeanor arrests among those who were diverted \parencite{perrone_harm_2022}. Although LEAD is not focused on OUD, the evidence suggests that utilizing police officers' role in society (e.g., broad mandate, gatekeepers of CJS) to divert low-level offenders away from the criminal justice system by creating pathways to behavioral and social services can reduce contact with the criminal justice system and improve employment and housing situations.

\subsubsection{Police Assisted Addiction Recovery Initiative (PAARI)}

- more specific to drugs, particularly opioids
Another program that is more specifically focused on OUD is the Police Assisted Addiction Recovery Initiative (PAARI). 

- mixture of diversion/referral : a focus on law enforcement as the mechanism
- examples/ evidence

\subsection{Police-led Partnerships}

- Yatsco (2020) and Formica (2018)
- recently more common
- limited empirical work

\section{\centering Current Study}
\section{\centering Methods}
\subsection{Setting}

\subsection{Data}
Data for this study come from three primary sources (TFMR data, Mesa open data, Census/ACS demographic estimates)

Due to data constraints in obtaining monthly opioid overdose data at the city-level, I opted to use Mesa's open data portal which provides calls for service to opioid overdoses. This is the only open data portal in the state of Arizona that provides such data (Besides Tempe). The CDC provides monthly data, however, only at the county and state levels. Other opioid overdose data repositories that were considered did not have monthly data either.
%\footnote{\url{https://drexel.edu/uhc/research/projects/BCHI/Drug%20Overdose%20Deaths%20in%20Big%20Cities/}}
%and \url{https://skylab.cdph.ca.gov/ODdash/?tab=Home}

\subsection{Dependent Variables}
\subsection{Independent Variable}
\subsection{Analytical Sample}
\subsection{Analytical Plan}

To evaluate the impact of the Tempe ORP, I use a Comparative Interrupted Time Series Design (CITS). This approach is employed for a few reasons. First, the interrupted time series design is a strong quasi-experimental modeling technique (CITES). It has been used to evaluate policy interventions in a variety disciplines including public health, epidemiology, and criminal justice. Second, this design allows for the researcher to account for baseline mean and trend of the outcome variable over time.

I use Mesa as a location-based comparison group, which is a city in Arizona that is adjacent to Tempe. The use of Mesa as a comparison group is important for a few reasons. First, the comparison group improves the rigor of the single-group ITS (Interrupted Time Series) analysis by reducing the impact of confounding variables \parencite{shadish_experimental_2002}. The effects of Covid-19, for instance, began within a month or so of the start of the Tempe ORP. Thus, a single-group ITS would be limited in providing reliable estimates of program impact. Second, using Mesa as a location-based comparison group helps reduce the impact of other state-level or county-level exogenous shocks that may have occurred over the course of the study period that would. Lastly, the use of a comparison group builds upon prior evaluative work on collaborative post-overdose outreach programs that used single-group forecasting models \parencite{donnelly_law_2022}.

\subsubsection{Segmented Generalized Least Squares (GLS) Regression Model}

\[Y = \beta_0 + \beta_1 TX + \beta_2 T + \beta_3 TS + \beta_4 POST + \beta_5 TXT + \beta_6 TXTS + \beta_7 TXPOST + \epsilon \]

Where \(Y\) is the outcome of interest (opioid overdose fatality rate per 100,000). \(\beta_0\) is the intercept of outcome \(Y\). \(\beta_1\) is the coefficient for a dummy treatment variable (= 1 if group is in the treatment group). \(\beta_2\) represents the coefficient for time, which is a continuous measure. \(\beta_3\) is the coefficient for time since the intervention (\(t\)= 1, 2, 3 ... \(T\) for months after treatment). \(\beta_4\) is the coefficient for the post-intervention period (= 1 if in the post-treatment time period).  \(\beta_5\) corresponds to the coefficient of an interaction between the treatment dummy and time. \(\beta_6\) is the coefficient for an interaction between the treatment dummy and time since the intervention took place. Lastly, the \(\beta_7\) coefficient corresponds to an interaction between the treatment and post-intervention dummy variables. Error is captured by \(\epsilon\).

The above regression model is estimated in Stata \parencite{statacorp_stata_2023} using the \textit{xtgls} command. The \textit{xtgls} command is used because of the panel structure of the data (e.g., cities nested within months) and the flexibility of the generalized model. Panel heteroskedasticity was found using the \textit{xttest3} command in Stata, therefore I estimate the regression model to account for panel heteroskedasticity. Then, using the \textit{xtserial} command I test for first-order autocorrelation but no first-order autocorrelation is detected.  

\section{\centering Results}

\section{\centering Discussion}
\subsection{Limitations}
\subsection{Implications}
\section{\centering Conclusion}
