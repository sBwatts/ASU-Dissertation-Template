\chapter{Collaborative Response to the Opioid Overdose Crisis: Evidence from a Quasi-Experimental Analysis}

\section{Introduction}
\section{Literature Review}
\subsection{}
\section{Methods}
\subsection{Setting}
\subsection{Data}
\subsection{Dependent Variables}
\subsection{Independent Variable}
\subsection{Analytical Sample}
\subsection{Analytical Plan}
\subsubsection{Segmented Generalized Least Squares (GLS) Regression Model}

\[Y = \beta_0 + \beta_1 TX + \beta_2 T + \beta_3 TS + \beta_4 POST + \beta_5 TXT + \beta_6 TXTS + \beta_7 TXPOST + \epsilon \]

Where \(Y\) is the outcome of interest (opioid overdose fatality rate per 100,000). \(\beta_0\) is the intercept of outcome \(Y\). \(\beta_1\) is the coefficient for a dummy treatment variable (=1 if group is in the treatment group). \(\beta_2\) represents the coefficient for time, which is a continuous measure. \(\beta_3\) is the coefficient for time since the intervention (=1 + \(t\) for months after treatment). \(\beta_4\) is the coefficient for the post-intervention period (=1 if in the post-treatment time period).  \(\beta_5\) corresponds to the coefficient of an interaction between the treatment dummy and time. \(\beta_6\) is the coefficient for an interaction between the treatment dummy and time since the intervention took place. Lastly, the \(\beta_7\) coefficient corresponds to an interaction between the treatment and post-intervention dummy variables. Error is captured by \(\epsilon\).

\section{Results}
